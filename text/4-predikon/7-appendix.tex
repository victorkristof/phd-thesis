\section{Experimental Setting}
\label{app:experimental_setting}

We describe in details the experimental setting and the choice of hyperparameters.
As explained in Section~\ref{sec:experiments}, we evaluate the predictive performance of our models and of the baselines after observing a fraction of regional outcomes.
For each dataset, we use the training set as validation set to find the best hyperparameters from a range of possible values.
We preserve the temporal order of the data, and we use the first $V$ votes for training while testing on the next vote $V+1$.
This simulates a real setting where we use all available data prior to an election or referendum of interest.

For each test vote, we simulate several random reveal orders of regional results.
For the Swiss referenda, the U.S.\ presidential election, and the German parliamentary election by district, we simulate reveal orders on a logarithmic space.
This emphasizes the importance of early results, \textit{i.e.}, when a small number of results are available, in the prediction performance.
For the German parliamentary election by states, the number of regions is small enough ($R=16$) to simulate reveal orders on a linear space.

We evaluate the performance of our models and of the baselines with different combinations of the hyperparameters.
The hyperparameters in our method are the $\ell_2$-regularization parameter $\lambda$ and the number of dimensions $D$ of the latent factors.
We report in Table~\ref{tab:ranges} the ranges of hyperparameters for each dataset and in Table~\ref{tab:best} the best combinations in terms of MAE.
In our experiments, we generally observed that the performance of an algorithm was robust to different combinations of hyperparameters.
In light of Occam's razor, we chose the simplest model, \textit{i.e.}, the lowest number of latent dimensions $D$ and the lowest value of regularization parameter $\lambda$, when different combinations reached equal values of MAE.

Finally, we measure the performance of an algorithm given some hyperparameters in terms of the MAE and the $\ell_1$-norm, as defined in Equation~\eqref{eq:mae} of Section~\ref{sec:experiments}.
We use the MAE (and the $\ell_1$-norm) as it measures the error in percentage points and provides, therefore, some interpretability.
For the binary datasets, \textit{i.e.}, for the Swiss referenda and for the U.S.\ presidential election, we measure the performance of a combination in terms of the MAE.
For the categorical datasets, \textit{i.e.}, for the German elections where we try to predict the fractions of votes that parties obtain, we measure the performance in terms of the $\ell_1$-norm to sum up the prediction errors across different parties.

\begin{table}
  \caption{
    Ranges of hyperparameters for different datasets.
  }
  \label{tab:ranges}
  \begin{tabular}{llrr}
    \toprule
    Country & Region & $\lambda$ & $D$ \\
    \midrule

    Switzerland & Munic.   & $\{0.001, 0.01, 0.1 \}$ & $\{10, 25, 100, 250\}$ \\
    U.S.        & State    & $\{0.001, 0.01, 0.1 \}$ & $\{3, 5, 7\}$ \\
    Germany     & State    & $\{0.001, 0.01, 0.1 \}$ & $\{3, 7, 11\}$ \\
    Germany     & District & $\{0.001, 0.01, 0.1 \}$ & $\{3, 7, 11\}$ \\

    \bottomrule
  \end{tabular}
\end{table}

\begin{table}
  \caption{
    Best hyperparameters for each model and dataset.
  }
  \label{tab:best}
  \begin{tabular}{lllrr}
    \toprule
    Country & Region & Likelihood & $\lambda$ & $D$ \\
    \midrule

    Switzerland & Munic.   & Gaussian  & $0.1$  & $25$ \\
                &          & Bernoulli & $0.01$ & $25$ \\
    U.S.        & State    & Gaussian  & $0.01$ & $5$ \\
                &          & Bernoulli & $0.01$ & $7$ \\
    Germany     & State    & Gaussian  & $0.01$ & $7$ \\
                &          & Bernoulli & $0.01$ & $7$ \\
    Germany     & District & Gaussian  & $0.01$ & $11$ \\
                &          & Bernoulli & $0.01$ & $11$ \\

    \bottomrule
  \end{tabular}
\end{table}

\subsection{Swiss Referenda}%
\label{app:ch}

In Switzerland, referenda occur when \numprint{50000} people petition against a law that has been accepted by the Parliament (optional referendum) or when the Swiss Constitution is modified (mandatory referendum).
Popular initiatives occur when \numprint{100000} people suggest a new law.
For simplicity, we refer to optional referenda, mandatory referenda, and popular initiatives as \textit{referenda}.

We collected the data about Swiss referenda between 1981 and 2020 from the Swiss Federal Statistical Office.
The data are published through an API on the Swiss Open Data platform\footnote{\url{https://opendata.swiss/en/dataset/echtzeitdaten-am-abstimmungstag-zu-eidgenoessischen-abstimmungsvorlagen}}
We pre-process the data as follows:
First, we remove 12 regions (\textit{i.e.}, the municipalities) with missing values.
This may happen because, each year, some municipalities are merged or split, and some results might not exist for some votes.
Second, we merge the regions that change their name, and we average their results.

In total, there are $V=326$ referenda and $R=2186$ municipalities in the data set used for the evaluation.
The validation set consists of referenda $275$ to $300$, and we use it to find the best hyperparameters.
The test set consists of referenda $301$ to $326$, and we use it to report the results in Section~\ref{sec:experiments}.
We test values for $\lambda \in \{0.001, 0.01, 0.1\}$ and for $D \in \{10, 25, 100, 250\}$.
The best model with Gaussian likelihood uses $\lambda=0.1$ and $D=25$ while the best model with Bernoulli likelihood uses $\lambda=0.01$ and $D=25$.
We tune the hyperparameters over $10$ random reveal orders per referendum, and we evaluate the performance of our algorithm over $100$ random reveal orders.
For the matrix factorization baseline, we use the best hyperparameters as reported by \citet{etter2016online}, \textit{i.e.}, $\lambda_U = 31.0$, $\lambda_V = 0.03$, and $D = 25$.

\subsection{US Presidential Election}%
\label{app:us}

The U.S.\ presidential election relies on the Electoral College system.
In this system, 538 delegates are assigned to each state proportionally to their population, and a candidate who obtains the majority of votes in a state wins all the delegates in that state\footnote{With the exception of Maine and Minnesota, which have a different rules.}.
The candidate who wins the majority of the delegates among all the states, \textit{i.e.}, at least 270 delegates, is elected president.
Because a candidate wins the same number of delegates whether it receives 99\% of the votes or 51\% of the votes, the collegial system leads to some unexpected behaviour: A candidate may win the popular vote but lose the collegial vote.
This happened only in two elections in our dataset: in 2000 and in 2016.
This special structure adds one level of complexity to the prediction task, and it requires further modeling assumptions.
To keep our approach general and because a mismatch between the popular vote and the collegial vote is rare, we keep this specificity of the U.S.\ electoral system for future work.

We use the U.S.\ presidential election data between 1976 and 2016.
The data is publicly available on Harvard Dataverse~\cite{mit2017us}.
The data reports state-level election outcomes with the number of votes received by each candidate.
We transform the outcome of the election into a binary outcome of Democrat candidate and Republican candidate.
Candidates from other parties are ignored, and we normalize the results of the candidates from the two major parties so that the sum of their votes is 1.
In comparison to the Swiss referenda, aggregating by number of voters will, therefore, be inexact for the U.S.\ presidential elections.
This is nonetheless a reasonable approximation, as the number of votes received by candidates from other parties are marginal compared to the candidates from the Republican party and from the Democrat party.

In total, there are $V=11$ elections.
The data include the results of the District of Columbia (\textit{i.e.}, Washington D.C.), which has a special status and is not considered a state; hence, we have $R=51$ regions, combining 50 states and the District of Columbia.
We find the best hyperparameters using the vote prior to the 2012 election, and we evaluate the model on the 2016 election.
We test values for $\lambda \in \{0.001, 0.01, 0.1 \}$ and for $D \in \{3, 5, 7\}$ as we only have 9 elections prior to 2012.
For both the Gaussian and Bernoulli likelihoods, the best model uses $\lambda = 0.01$.
The best model with Gaussian likelihood uses $D=5$ and the best model with Bernoulli likelihood uses $D=7$.
We tune the hyperparameters over $100$ random reveal orders per election, and we evaluate the performance of our algorithm over $10000$ random reveal orders.

\subsection{German Parliamentary Election}%
\label{app:ge}

We use the German parliamentary elections data published by the European Elections Database (EED) \cite{norsk2020germany}.
In Germany, parliamentary elections take place every 4 years.
The EED reports results between 1990 and 2009 on state level and between 1990 and 2005 on district level.
Similarly to the U.S.\ presidential elections, we normalize the results per region by keeping the main five parties in Germany (CDU/CSU, SPD, FDP, the Greens, and the Left).

In total, there are $V=6$ state-level elections and $V=5$ district-level elections.
For state-level elections, we find the best hyperparameters using the votes prior to the 2005 elections and we evaluate the model on the 2009 elections.
For district-level elections, we find the best hyperparameters using the votes prior to the 2002 elections and we evaluate the model on the 205 elections.
We test values for $\lambda \in \{0.001, 0.01, 0.1 \}$ and for $D \in \{3, 7, 11\}$.
For both datasets, both the Gaussian and the Bernoulli likelihoods provide the same results.
For state-level elections, the best model uses $\lambda=0.01$ and $D=7$.
For district-level elections, the best model uses $\lambda=0.01$ and $D=11$.
In both cases, we tune the hyperparameters over $100$ random reveal orders per election, and we evaluate the performance of our algorithm over $1000$ random reveal orders.
