\section{Conclusion}%
\label{sec:conclusion}

We have proposed an algorithm to predict national vote results from regional results that are observed sequentially.
Our approach learns a representation for each region by factorizing the sub-matrix of historical data and approximating the representation of a new vote as the optimal parameters of a generalized linear model.
The predictions for unobserved results are obtained through the link function of the GLM, and national predictions are obtained by aggregating observed and unobserved regional results.
We are able to predict both referenda with binary outcomes and elections with categorical outcomes.
We have shown that our approach outperforms the (weighted) average of partial results on three datasets of Swiss referenda, U.S.\ presidential elections, and German legislative elections.
We have explored the regional representations in their latent space and have shown that they capture ideological and cultural patterns.
Finally, we have deployed a Web platform to provide real-time vote predictions for Swiss referenda.
Our algorithm is able to predict the final outcome of four real votes with an absolute error of about 1\% after observing only 13\% of the ballots.

\paragraph{Future Work}

We plan to further develop our approach in three directions.
First, Bayesian inference in our generalized linear model would enable uncertainty quantification of our predictions in a principled way.
This could be beneficial for predictions, especially during the early counting phase.
Bayesian inference for GLMs has been widely studied in the literature~\citep{murphy2012machine}.
Second, our algorithm is capable of making predictions only with at least one observed regional result.
In the spirit of~\citet{etter2016online}, we plan to augment our algorithm with features from the vote and the municipalities to make predictions prior to referenda in Switzerland.
One limitation of their work lies in the lack of systematic availability of the features they include in their model.
In particular, every Swiss citizen receives documentation about each referendum.
These explanatory documents provide a valuable source of information about a vote, one that could be incorporated in a predictive model.
The actual text of the proposed laws would provide another source of relevant information.
Finally, by collecting the sequential order by which regional results arrive in Swiss referenda, we obtain data about the true reveal order.
We plan to explore whether the true sequential order can be exploited to learn the schedule by which results arrive and, therefore, further improve the earliest predictions.
