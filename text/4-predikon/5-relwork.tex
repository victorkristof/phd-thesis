%! TEX root = ../thesis.tex
\section{Related Work}%
\label{pdk:sec:relwork}

We base the present paper on the work of~\citet{etter2014mining} and build on their approach proposed in~\cite{etter2016online}.
They combine matrix factorization and Gaussian processes (GP) to understand what features of the votes and of the municipalities contribute the most to the predictive performance.
They develop an expectation-maximization algorithm to learn both latent features and the GP parameters jointly.
They show that the geographical location of municipalities is the most important feature for making predictions, an aspect that is in part captured by the feature matrix $\vX$ of Equation~\eqref{pdk:eq:projection} in our algorithm and illustrated in Figure~\ref{pdk:fig:ch_svd}:
Municipalities that are geographically close tend to speak the same language.
They also show that they are able to make accurate predictions of Swiss referenda.
In comparison, our method is more efficient, as it learns the latent features of municipalities $\vX$ through singular value decomposition offline, and it learns the latent features of a vote through a GLM.
The GLM also provides more flexibility:
Our algorithm could conceivably be used to make prediction for other types of observations, \textit{e.g.}, count data, and works for non-binary outcomes.
We developed our algorithm with applicability in mind.
Our main goal was to make real-time predictions for Swiss referenda, with all the constraints that come with this problem.

The problem we address, \textit{i.e.}, predicting unobserved entries of a new column of a matrix from partial observations of that column, is most similar to the problem of missing-data imputation.
The use of SVD for data imputation has been studied in the context of genomics~\citep{troyanskaya2001missing,hastie1999imputing}.
In gene matrices, missing entries are common, and the authors propose an algorithm based on SVD to impute missing data.
Their algorithm iteratively computes the SVD of an approximation to the full matrix and predicts the missing values with a regression by using the non-missing values to refine the approximation.
An extensive literature review of predictive methods for data imputation is available in~\citet{bertsimas2017predictive}.
Incremental SVD revisions have been studied in the context of computer vision~\citep{brand2002incremental} and recommender systems~\citep{brand2003fast}.
In this latter work, the author proposes algorithms to compute the SVD of a matrix when new columns arrive sequentially and are corrupted by some noise (\textit{e.g.}, some entries are missing).
Their solution is equivalent to our \textsc{SubSVD-Gaussian} algorithm without regularization, \textit{i.e.}, $\lambda=0$, for which a closed form solution is provided in Equation~\eqref{pdk:eq:least_squares}.

A whole body of work in the political science community exists on election forecasting~\citep{lewis2005election}, \textit{i.e.}, predicting the outcome of an election before it happens.
The seminal work of~\citet{bean1948predict}, who first studied this problem in 1948, looked at using historical data to find U.S.\ states that were the most predictive of the national outcome.
Statistical models for election forecasting have since been developed in many contexts for Germany~\citep{tumasjan2011election}, France~\citep{belanger2004finding}, the U.K.~\citep{franch2013wisdom}, and the U.S.~\citep{rigdon2009bayesian,kennedy2017improving}.
The prediction of U.S.\ elections has been popularized by the blogger and statistician Nate Silver in 2008 as he predicted Barack Obama's victory in the Democratic Party primaries using a statistical model of historical data~\citep{blumenthal2008poblano}, and as he predicted Barack Obama's victory in the presidential election from polling data~\citep{silver2008pollster}.
In the computer science community, algorithms for election forecasting have also been developed using social media data in Denmark~\citep{kristensen2017parsimonious}, Finland~\citep{vepsalainen2017facebook}, the U.S.~\citep{choy2012us,ramteke2016election}, and the developing world~\citep{dwi2015twitter}.
To the best of our knowledge, except for the work mentioned at the beginning of this section, we are the first to study real-time outcome predictions of elections and referenda, and to deploy a system for making predictions of Swiss referenda in real-time.
