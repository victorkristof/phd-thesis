%! TEX root = ../thesis.tex

In this chapter\footnote{This chapter is based on \citet{immer2020sub}.}, we address the problem of predicting aggregate vote outcomes (\textit{e.g.}, national) from partial outcomes (\textit{e.g.}, regional) that are revealed sequentially.
We combine matrix factorization techniques and generalized linear models (GLMs) to obtain a flexible, efficient, and accurate algorithm.
While our approach does not use discrete-choice models directly, the problem we tackle is related to one of the most fundamental choice processes: voting.
Our algorithm works in two stages:
First, it learns representations of the regions from high-dimensional historical data.
Second, it uses these representations to fit a GLM to the partially observed results and to predict unobserved results.
We show experimentally that our algorithm is able to accurately predict the outcomes of Swiss referenda, U.S.\ presidential elections, and German legislative elections.
We also explore the regional representations in terms of ideological and cultural patterns.
Finally, we deploy an online Web platform\footnote{\href{https://www.predikon.ch}{https://www.predikon.ch}} to provide real-time vote predictions in Switzerland and a data visualization tool to explore voting behavior.
