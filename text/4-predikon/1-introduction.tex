%! TEX root = ../thesis.tex
\section{Introduction}%
\label{pdk:sec:introduction}

% General introduction.
The past decade has seen the emergence of several open-government initiatives for the increase of administration transparency through the publication of governmental data.
These data are of great interest to parties, companies, sub- and supra-government entities, researchers, and citizens.
In particular, the results of referenda and election ballots in municipalities, districts, states, and countries are valuable for understanding the structure and the dynamics of politics.

% Describe the interest by media, parties, interest groups, and people of having an early prediction.
In this chapter, we address the problem of vote prediction when only partial results are available.
The ability to predict the outcome of votes both before and during ballot counting is relevant to political parties, interest groups, polling agencies, news outlets, government authorities, and interested citizens.
These predictions help uncover voting patterns, \textit{e.g.}, to identify swing regions, to understand voting behaviours, and to detect fraud.
Political parties and interest groups can enhance their campaigning efforts.
Polling agencies and news outlets can optimize their surveying efforts.
Authorities can monitor the smooth functioning of the voting process.

% Describe the main problem we are solving.
We focus on national vote predictions during the ballot counting, \textit{i.e.}, after all eligible voters have cast their ballots, as government officials start count the valid votes in each region.
We predict national results by using sequential regional results, and we seek to obtain accurate predictions as early as possible, \textit{i.e.}, with a minimum number of regional results.
Typically, less populated regions release their official counts earlier than more populated ones.
Regions where remote voting is allowed release their results earlier than regions where this is not allowed.
In some countries, for example in the U.S., some regions vote earlier than others by design.
We will show that our model is able to exploit the correlations between regions and between votes to obtain accurate early predictions.

% Explain that we focus on Switzerland and why.
Switzerland offers a fascinating laboratory for vote prediction due to its direct-democracy system.
Swiss citizens are called to vote four times a year on referenda and popular initiatives~\citep{confederation2019democracy, confederation2019popular}.
As a result, the amount and frequency of voting data produced in Switzerland remains unmatched by any other country.
We take Switzerland as an example to develop our methodology but, as shown in Section~\ref{pdk:sec:experiments}, our algorithm can be applied to other countries and in other settings.

% Overview of our model for vote outcome.
In Section~\ref{pdk:sec:methodology}, we propose an algorithm to predict national vote outcomes from a sequence of regional vote results.
Our model has two components:
First, it learns the correlations between regions and between votes from historical data by using singular value decomposition (SVD).
Second, after observing at least one regional result for a new vote, it uses the SVD as input features to a generalized linear model (GLM) to predict the unobserved regional results.
The national outcome is then easily obtained by weighted aggregation of the predicted and the observed regional results.
The SVD, computed only once on the historical data, is inexpensive in terms of complexity and enables interpretation.
By using different likelihoods in the GLM, we gain flexibility in predicting binary outcomes (for votes) or categorical outcomes (for elections).

% Our model can be generalized to elections and mention US and Germany as an example.
For Swiss votes, where people must answer "Yes" or "No" on each ballot, we show that a Gaussian and a Bernoulli likelihood provide the best performance.
We also explore what the SVD offers in terms of interpretation of voting patterns.
Furthermore, we show that we can predict the outcome of the popular vote of a U.S.\ presidential election by casting this problem as a binary choice between two candidates.
We predict the outcome of parliamentary elections in Germany, where people must choose between five political parties, using a categorical likelihood.
We describe our experiments on state-level and district-level results in Section~\ref{pdk:sec:experiments}.

% Mention Predikon.ch.
We also deploy a Web platform available to the general public to provide vote prediction for Switzerland.
Using an API developed by the Swiss government, we are able to make real-time predictions during the official counting with partial regional results.
We also provide a data-visualization tool to explore voting patterns and to understand how our model makes predictions.
We describe our platform in Section~\ref{pdk:sec:depsys}.

% Our contributions.
In summary, our contributions are as follows:
We propose an efficient, flexible, and accurate algorithm for predicting the national outcome of a referendum or an election from early regional results.
We curate a new dataset of sequential vote results in Switzerland, covering \numprint{330} votes and \numprint{2196} regions between 1981 and 2020.
We deploy an interactive Web platform to display real-time vote predictions in Switzerland, together with tools to explore and visualize our dataset.
The data and the code are available on \href{https://www.github.com/indy-lab/submatrix-factorization}{github.com/indy-lab/submatrix-factorization} and the Web platform is available on \href{http://www.predikon.ch}{www.predikon.ch}.
