\chapter{Conclusion}
\label{ch:conclusion}

Modern technologies enable the collection of comparison data at an unprecedented scale, opening up many new opportunities for businesses and researchers.
But they also raise substantial challenges.
Often, the number of parameters of the models used to analyze the data grows concurrently to the amount of data.
This calls for new, efficient methods for collecting comparisons and learning models.
In this thesis, we propose several solutions that highlight different aspects of efficiency.

\begin{itemize}
\item In Chapter~\ref{ch:fastinference}, we address the problem of parameter inference for Luce's choice model.
By expressing stationary points of the likelihood function as the stationary distribution of a Markov chain, we link recently proposed spectral estimators to maximum-likelihood methods.
This link enables the development of new inference algorithms that are statistically and computationally efficient.

\item In Chapter~\ref{ch:robustsort}, we consider the active-learning setting.
We study theoretically and empirically the performance of Quicksort when pairwise comparison outcomes follow the Bradley--Terry model.
In scenarios where it is possible to adaptively select pairs of items to compare, we show that sorting-based active-learning strategies lead to significant gains in sample efficiency.
Compared to competing active-learning methods, ours is computationally cheaper.

\item In Chapter~\ref{ch:choicerank}, we focus on choices in networks.
In this case, we achieve data efficiency in a different way: we find that it is not necessary to observe distinct choices among well-defined sets of alternatives in order to estimate model parameters.
Marginal information about the incoming and outgoing traffic at each node is sufficient.
The network structure also enables a fast algorithm that scales to very large graphs.

\item Finally, in Chapter~\ref{ch:playerkern}, we tackle a concrete problem in sports.
Based on past outcomes of football matches, we seek to predict the outcome of future matches between national teams.
We devise a method that uses all the available data efficiently: it considers the outcome of all matches---including those between clubs---and obtains predictions that outperform competing models.
It does so by implicitly projecting the football matches in the space of players.
\end{itemize}

The approach we take in most of this thesis consists of distilling challenges faced in modern applications of choice models into simple and fundamental problems.
We then propose methods to address these problems.
We have applied these methods to real use-cases; however, there remain important classes of practical applications for which our methods are not applicable directly.
We discuss three directions in which our work could be extended.

\begin{description}
\item[Item features] With the exception of Chapter~\ref{ch:playerkern}, we have assumed that the item strengths $\{ \gamma_i \}$ or $\{ \theta_i \}$ are free parameters.
However, in some applications, we might have access to features that relate items to each other.
We distinguish two cases.
First, suppose that item $i$ is described by a real-valued feature vector $\bm{x}_i \in \mathbf{R}^D$, where typically $D \ll N$.
Then, by setting $\theta_i = \bm{x}_i^\Tr \bm{w}$ for some latent parameter vector $\bm{w} \in \mathbf{R}^D$, we obtain the multinomial logit model \citep{mcfadden1973conditional, train2009discrete}.
Inference in this model is well-studied, but the issue of effective and efficient active learning remains widely open.
Second, suppose that item $i$ is described by a binary vector $\bm{x}_i \in \{0, 1\}^D$ describing the presence or absence of certain features.
Then, we can model comparison outcomes using the elimination-by-aspects (EBA) model \citep{tversky1972elimination}, a model closely related to that of Luce.
Preliminary work shows that the algorithms developed in Chapter~\ref{ch:fastinference} could be extended to the EBA model in the case of pairwise comparisons.

\item[Context of comparisons] Sometimes, the context in which choices are made is important.
For example, we might prefer to listen to a different type of music depending on whether we are spending a quiet moment or doing sports, or whether it is summer or Christmas time, etc.
In cases where the context is explicit, we fall back to the problem of integrating side information in the form of feature vectors.
However, if the context is not explicitly observed, the problem becomes more difficult.
We make a step towards addressing this problem in \citet{ko2016collaborative}, where we study a setting in which we observe sequences of choices.
We propose a model where the context at time $t$ is encoded by previous choices made by a user in $(-\infty, t)$, and where the effective utility $\theta_i^{(t)}$ of item $i$ at time $t$ varies accordingly.
In general, integrating latent context into choice models remains an interesting avenue for future research.

\item[Personalization] In applications using recommender systems, the task is often to learn a \emph{distinct} preference profile for each user.
For example, online service providers use these systems in order to tailor their service to the specific tastes of a user.
An obvious but inefficient way to achieve personalization is to learn a distinct choice model for every user.
As many users share similar preferences, it is sensible to take advantage of similar users' choices to learn a given user's preferences.
One approach is to postulate that there are a small number of (global) instances of Luce's model, and that individual preferences are formed by (user-specific) mixtures of these models \citep{gormley2008exploring, ammar2015ranked}.
The inference problem then consists of jointly learning the global models and the user-specific mixture weights.
To this end, the algorithms developed in Chapter~\ref{ch:fastinference} could be used to carry out the \emph{M} step in the \emph{EM} algorithm of \citet{gormley2008exploring}.
\end{description}

A potential weakness of models based on Luce's axiom (as well as those based on Thurstone's ideas, see Section~\ref{in:sec:thurstone}) is that they are sensitive to outliers: the probability that the outcome of a comparison between $i$ and $j$ is inconsistent decreases \emph{exponentially} fast with $\Abs{\theta_i - \theta_j}$.
Hence, a small fraction of outliers (due, e.g., to the actions of dishonest users) can affect the model significantly.
In addition to the extensions outlined above, the study of robust alternatives to Luce's model is an important direction for future work.
On the one hand, model inference will likely require the development of new tools, beyond those presented in this thesis.
On the other hand, there is hope that the sorting-based active-learning strategies presented in Chapter~\ref{ch:robustsort}---including the theoretical bounds on Quicksort's performance---can be extended to heavy-tailed noise.

In conclusion, we hope to have convinced the reader that the study of comparison models is of paramount importance to improve online applications, because choices are the most natural way for humans to express their opinions (whether implicitly or explicitly).
This thesis hopefully brings us a step closer towards effective methods for eliciting and analyzing comparison outcomes.
As several challenges remain, research on choice models has a bright future ahead of itself.
