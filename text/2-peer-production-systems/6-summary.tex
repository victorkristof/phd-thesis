%! TEX root = ../thesis.tex
\section{Summary}
\label{pps:sec:conclusion}

In this paper, we have introduced \interank{}, a model of edit outcomes in peer-production systems.
Predictions generated by our model can be used to prioritize the work of project maintainers by identifying contributions that are of high or low quality.

Similarly to user reputation systems, \interank{} is simple, easy to interpret and applicable to a wide range of domains.
Whereas user reputation systems are usually not competitive with specialized edit quality predictors tailored to a particular peer-production system, \interank{} is able to bridge the gap between the two types of approaches, and it attains a predictive performance that is competitive with the state of the art---without access to content-based features.

We have demonstrated the performance of the model on two peer-production systems exhibiting different characteristics.
Beyond predictive performance, we can also use model parameters to gain insight into the system.
On Wikipedia, we have shown that the model identifies controversial articles, and that latent dimensions learned by our model display interesting patterns related to cultural distinctions between articles.
On the Linux kernel, we have shown that inspecting model parameters enables to identify core subsystems (large difficulty parameters) from peripheral components (small difficulty parameters).

\paragraph{Perspective}
One direction to explore is the idea of using the latent embeddings learned by our model in order to recommend items to edit.
Ideally, we could match items that need to be edited with users that are most suitable for the task.
For Wikipedia, an ad-hoc method called ``SuggestBot'' was proposed by~\citet{cosley2007suggestbot}.
We believe it would be valuable to propose a method that is applicable to peer-production systems in general.
