\section{Statistical Models}
\label{sec:models}

In this section, we describe and explain two variants of a statistical model of edit outcomes based on \emph{who} edits \emph{what}.
In other words, we develop models that are predictive of the outcome $q \in \{0, 1\}$ of a contribution of user $u$ on item $i$.
To this end, we represent the probability $p_{ui}$ that an edit made by user $u$ on item $i$ is successful.
In collaborative projects of interest, most users typically interact with only a small number of items.
In order to deal with the sparsity of interactions, we postulate that the probabilities $\{ p_{ui} \}$ lie on a low-dimensional manifold and propose two model variants of increasing complexity.
In both cases, the parameters of the model have intuitive effects and can be interpreted easily.

\paragraph{Basic Variant}
The first variant of our model is directly inspired by the Rasch model \citep{rasch1960probabilistic}.
The probability that an edit is accepted is defined as
\begin{align}
\label{eq:basicmodel}
p_{ui} = \frac{1}{1 + \exp[-(s_u - d_i + b)]},
\end{align}
where $s_u\in\mathbf{R}$ is the \emph{skill} of user $u$, $d_i\in\mathbf{R}$ is the \emph{difficulty} of item $i$, and $b \in \mathbf{R}$ is a global parameter that encodes the overall skew of the distribution of outcomes.
We call this model variant \interank{basic}.
Intuitively, the model predicts the outcome of a ``game'' between an item with inertia and a user who would like to effect change.
The \emph{skill} quantifies the ability of the user to enforce a contribution, whereas the \emph{difficulty} quantifies how ``resistant'' to contributions the particular item is.
%An increasing item difficulty $d_i$ monotonically decreases the probability $p_{ui}$, whereas an increasing user skill $s_u$ monotonically increases $p_{ui}$.
%This corresponds to an appropriate first-order approximation to the outcome probability.

Similarly to reputation systems \citep{adler2007content}, \interank{basic} learns a score for each user; this score is predictive of edit quality.
However, unlike these systems, our model also takes into account that some items might be more challenging to edit than others.
For example, on Wikipedia, we can expect high-traffic, controversial articles to be more difficult to edit than less popular articles.
As with user skills, the article difficulty can be inferred \emph{automatically} from observed outcomes.

\paragraph{Full Variant}
Although the \emph{basic} variant is conceptually attractive, it might prove to be too simplistic in some instances.
In particular, the \emph{basic} variant implies that if user $u$ is more skilled than user $v$, then $p_{ui} > p_{vi}$ for \emph{all} items $i$.
In many peer-production systems, users tend to have their own specializations and interests, and each item in the project might require a particular mix of skills.
For example, with the Linux kernel, an engineer specialized in file systems might be successful in editing a certain subset of software components, but might be less proficient in contributing to, say, network drivers, whereas the situation might be exactly the opposite for another engineer.
In order to capture the multidimensional interaction between users and items, we add a bilinear term to the probability model~\eqref{eq:basicmodel}.
Letting $\bm{x}_u, \bm{y}_i \in \mathbf{R}^D$ for some dimensionality $D \in \mathbf{N}_{>0}$, we define
\begin{align}
\label{eq:fullmodel}
p_{ui} = \frac{1}{1 + \exp[-(s_u - d_i + \bm{x}_u^\top \bm{y}_i + b)]}.
\end{align}
We call the corresponding model variant \interank{full}.
The vectors $\bm{x}_u$ and $\bm{y}_i$ can be thought of as embedding users and items as points in a latent $D$-dimensional space.
Informally, $p_{ui}$ increases if the two points representing a user and an item are close to each other, and it decreases if they are far from each other (e.g., if the vectors have opposite signs).
If we slightly oversimplify, the parameter $\bm{y}_i$ can be interpreted as describing the set of skills needed to successfully edit item $i$, whereas $\bm{x}_u$ describes the set of skills displayed by user~$u$.

The bilinear term is reminiscent of matrix-factorization approaches in recommender systems \citep{koren2009matrix};
indeed, this variant can be seen as a \emph{collaborative-filtering} method.
In true collaborative-filtering fashion, our model is able to learn the latent feature vectors $\{ \bm{x}_i \}$ and $\{ \bm{y}_i \}$ \emph{jointly}, by taking into consideration all edits and without any additional content-based features.

Finally, note that the skill and difficulty parameters are retained in this variant and can still be used to explain first-order effects.
The bilinear term explains only the additional effect due to the user-item interaction.

\subsection{Learning the Model}
\label{sec:learning}

From~\eqref{eq:basicmodel} and~\eqref{eq:fullmodel}, it should be clear that our probabilistic model assumes no data other than the identity of the user and that of the item.
This makes it generally applicable to any peer-production system in which users contribute to discrete items.

Given a dataset of $K$ independent observations $\mathcal{D} = \{ (u_k, i_k, q_k) \mid k = 1, \ldots, K \}$, we infer the parameters of the model by maximizing their likelihood under $\mathcal{D}$.
That is, collecting all model parameters into a single vector $\bm{\theta}$, we seek to minimize the negative log-likelihood
\begin{align}
\label{eq:nll}
- \ell (\bm{\theta} ; \mathcal{D}) = \sum_{(u,i,q) \in \mathcal{D}} \left[ -q \log p_{ui} - (1 - q) \log (1 - p_{ui}) \right],
\end{align}
where $p_{ui}$ depends on $\bm{\theta}$.
In the \emph{basic} variant, the negative log-likelihood is convex, and we can easily find a global maximum by using standard methods from convex optimization.
In the \emph{full} variant, the bilinear term breaks the convexity of the objective function, and we can no longer guarantee that we will find parameters that are global minimizers.
In practice, we do not observe any convergence issues but reliably find good model parameters on all datasets.

Note that~\eqref{eq:nll} easily generalizes from binary outcomes ($q \in \{0, 1\}$) to continuous-valued outcomes ($q \in [0, 1]$).
Continuous values can be used to represent the \emph{fraction} of the edit that is successful.

\paragraph{Implementation}
We implement the models in Python by using the TensorFlow library \citep{abadi2016tensorflow}.
Our code is publicly available online at \url{https://github.com/lca4/interank}.
In order to avoid overfitting the model to the training data, we add a small amount of $\ell_2$ regularization to the negative log-likelihood.
We minimize the negative log-likelihood by using stochastic gradient descent \citep{bishop2006pattern} with small batches of data.
For \interank{full}, we set the number of latent dimensions to $D = 20$ by cross-validation.

\paragraph{Running Time}
Our largest experiment consists of learning the parameters of \interank{full} on the entire history of the French Wikipedia (c.f. Section~\ref{sec:wikipedia}), consisting of over \num{65} million edits by \num{5} million users on \num{2} million items.
In this case, our TensorFlow implementation takes approximately \num{2} hours to converge on a single machine.
In most other experiments, our implementation takes only a few minutes to converge.
This demonstrates that our model effortlessly scales, even to the largest peer-production systems.

\subsection{Applicability}
Our approach models the difficulty of effecting change through the affected item's identity.
As such, it applies particularly well to peer-production systems where users \emph{cooperate} to improve the project, i.e., where each edit is judged independently against an item's (latent) quality standards.
This model is appropriate for a wide variety of projects, ranging from online knowledge bases (such as Wikipedia, c.f. Section~\ref{sec:wikipedia}) to open source software (such as the Linux kernel project, c.f. Section~\ref{sec:linux}).
In some peer-production systems, however, the contributions of different users \emph{compete} against each other, such as multiple answers to a single question on a Q\&A platform.
In these cases, our model can still be applied, but fails to capture the fact that edit outcomes are interdependent.
