%! TEX root = ../thesis.tex
% Max 250 wrods, c.f. https://wordcounter.net/.

In this chapter\footnote{This chapter is based on~\citet{yardim2018can}.}, we develop a discrete-choice model inspired from the Rasch model and including ideas reminiscent of collaborative filtering to predict user contributions to online peer-production systems.
As the number of contributors to these systems grows, it becomes increasingly important to predict whether the edits that users make will eventually be beneficial to the project.
Existing solutions either rely on a user reputation system or consist of a highly specialized predictor that is tailored to a specific peer-production system.
We explore a different point in the solution space that goes beyond user reputation but does not involve any content-based feature of the edits.
We posit that the probability that an edit is accepted is a function of the editor's skill, of the difficulty of editing the component and of a user-component interaction term.
Our model is broadly applicable, as it only requires observing data about \emph{who} makes an edit, \emph{what} the edit affects and whether the edit survives or not.
We apply our model on Wikipedia and the Linux kernel, two examples of large-scale peer-production systems, and we seek to understand whether it can effectively predict edit survival:
in both cases, we provide a positive answer.
Our approach significantly outperforms those based solely on user reputation and bridges the gap with specialized predictors that use content-based features.
It is simple to implement, computationally inexpensive, and in addition it enables us to discover interesting structure in the data.

%We apply it to Wikipedia and the Linux kernel, two examples of large-scale collaborative projects.
%In both cases, the model enables us (a) to discover interesting structure in the data and (b) to effectively predict whether edits will survive.
