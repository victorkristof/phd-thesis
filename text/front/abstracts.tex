%! TEX root = ../thesis.tex
% English abstract
\cleardoublepage
\chapter*{Abstract}
\markboth{Abstract}{Abstract}
\addcontentsline{toc}{chapter}{Abstract / Résumé} % adds an entry to the table of contents

Poor decisions and selfish behaviors give rise to seemingly intractable global problems, such as the lack of transparency in democratic processes, the spread of conspiracy theories, and the rise in greenhouse gas emissions.
However, people are more predictable than we think, and with machine-learning algorithms and sufficiently large datasets, we can design accurate models of human behavior in a variety of settings.
In this thesis, to gain insight into social processes, we develop highly interpretable probabilistic choice-models.
We draw from the econometrics literature on discrete-choice models and combine them with matrix factorization methods, Bayesian statistics, and generalized linear models.
These predictive models enable interpretability through their learned parameters and latent factors.

First, we study the social dynamics behind group collaborations for the collective creation of content, such as in Wikipedia, the Linux kernel, and the European Union law-making process.
By combining the Bradley-Terry and Rasch models with matrix factorization and natural language processing, we develop a model of edit acceptance in peer-production systems.
We discover controversial components (\textit{e.g.},~Wikipedia articles and European laws) and influential users (\textit{e.g.},~Wikipedia editors and parliamentarians), as well as features that correlate with a high probability of edit acceptance.
The latent representations capture non-linear interactions between components and users, and they cluster well into different topics (\textit{e.g.}, historical figures and TV characters in Wikipedia, business and environment in European laws).

Second, we develop an algorithm for predicting the outcome of elections and of referenda by combining matrix factorization and generalized linear models.
Our algorithm learns representations of votes and regions, which capture ideological and cultural voting patterns (\textit{e.g.}, liberal/conservative, rural/urban), and it predicts the vote results in unobserved regions from partial observations.
We test our model on voting data in Germany, Switzerland, and the US, and we deploy it on a Web platform to predict Swiss referendum votes in real-time.
On average, our predictions reach a mean absolute error of 1\% after observing only~5\% of the regions.

Third, we study how people perceive the carbon footprint of their day-to-day actions.
We cast this problem as a comparison problem between pairs of actions (\textit{e.g.}, the difference between flying across continents and using household appliances), and we develop a statistical model of relative comparisons reminiscent of the Thurstone model in psychometrics.
The model learns the users’ perception as the parameters of a Bayesian linear regression, which enables us to derive an active-learning algorithm to collect data efficiently.
Our experiments show that users overestimate the emissions of low-footprint actions and underestimate those of high-footprint actions.

Finally, we design a probabilistic model of pairwise-comparison outcomes that capture a wide range of time dynamics.
We achieve this by replacing the static parameters of a class of popular pairwise-comparison models with continuous-time Gaussian processes.
We also develop an efficient inference algorithm that computes, with only a few linear-time iterations over the data, an approximate Bayesian posterior distribution.

\paragraph{Keywords}
discrete-choice models, matrix factorization, Bayesian statistics, generalized linear models, comparisons, choices, probabilistic models, data mining, machine learning, computational social science

\cleardoublepage

% French abstract
\begin{otherlanguage}{french}
	\chapter*{Résumé}
	\markboth{Résumé}{Résumé}

	Les problèmes globaux, tels que le manque de transparence des processus démocratiques, la propagation de théories conspirationnistes ou l'augmentation des gaz à effet de serre, peuvent paraître imprévisibles et insolubles.
	Par contre, les êtres humains sont--heureusement--plus prévisibles que l'on ne pense.
	Grâce à des jeux de données massifs et de puissants algorithmes d'apprentissage automatique, il devient possible de modéliser une multitude de comportements sociaux.
	Dans cette thèse, nous développons des modèles probabilistes de choix individuels afin d'analyser ces comportements.
	Nous puisons dans la littérature des modèles de choix discrets, forts utilisés en économétrie, afin de rendre nos modèles interprétables, et nous les combinons avec des méthodes computationnelles, telles que la factorisation matricielle, les statistiques bayésiennes et les modèles linéaires généralisés, afin de les rendre plus performants.

	Premièrement, nous étudions la dynamique des systèmes collaboratifs de création de contenu, tels que Wikipédia, le système d'opération Linux, et les lois du Parlement européen.
	Nous combinons les modèles de Bradley-Terry et de Rasch en un nouveau modèle qui nous permet de prédire si les modifications de contenu sont acceptées ou non (par la communauté Wikipédia ou par les autres parlementaires, par exemple).
	Ce modèle révèle quels sont les composants importants de ces systèmes, tels que les articles de Wikipédia controversés ou les parlementaires influents, ainsi que les facteurs qui augmentent la probabilité qu'une modification soit acceptée.
	Notre modèle inclut également des facteurs latents qui améliorent les performances de prédictions.
	% Ces facteurs permettent aussi d'effectuer une analyse par regroupement, laissant émerger les sujets qui divisent le plus les utilisateurs (c'est-à-dire qu'ils sont soit connaisseurs d'un sujet, soit d'un autre, mais pas des deux).
	% Sur Wikipédia, par exemple, ce sont les personnages historiques d'un côté et la culture populaire de l'autre.

	Deuxièmement, nous développons un algorithme de prédiction des résultats du vote populaire d'élections et de référendums à partir d'observations régionales partielles.
	Notre approche combine la factorisation matricielle et les modèles linéaires généralisés afin d'apprendre des représentations vectorielles des votes et des régions.
	Ces représentations capturent les biais d'influence, comme les biais culturels, linguistiques ou idéologiques.
	Nous appliquons notre modèle à des données de vote pour l'Allemagne, les États-Unis et la Suisse, et nous le déployons sur une plateforme en ligne pour prédire les votations suisses en temps réel.
	En moyenne, nos prédictions sont correctes à moins de 1\% d'erreur en utilisant les résultats de seulement 5\% des communes.

	Troisièmement, nous nous intéressons à la perception que les gens ont de leur empreinte carbone.
	Nous formalisons ce problème sous forme de comparaisons entre deux actions (par exemple, prendre l'avion et utiliser un séchoir) et développons un modèle inspiré par l'approche de Thurstone en psychométrie.
	Le modèle apprend la perception générale d'une population d'individus en estimant les paramètres d'une régression linéaire bayésienne.
	Nos expériences montrent que les individus ont tendance à sur-estimer les actions à faible empreinte carbone et sous-estimer les actions à forte empreinte.

	Finalement, nous développons un modèle probabiliste dynamique de comparaison par paires.
	Nous remplaçons les paramètres statiques d'une famille de modèles de comparaison par des processus gaussiens à temps continu.
	Nous développons également un algorithme d'inférence qui calcule une approximation bayésienne de la distribution postérieure du modèle de manière efficace, en quelques itérations à temps linéaire sur les données.

	\paragraph{Mots-clés}
	modèles de choix discrets, factorisation matricielle, statistiques bayésiennes, modèles linéaires généralisés, comparaisons, choix, modèles probabilistes, analyse de données, apprentissage automatique, sciences sociales computationnelles
\end{otherlanguage}
