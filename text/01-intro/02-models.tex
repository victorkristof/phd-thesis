\section{Probabilistic Models of Choice}
\label{in:sec:models}

\subsection{A Brief History of Choice Models}
% Why studying choices to study human behaviour is important and useful
% - Long literature from psychometrics and econometrics
% - New methods enabled by computer science to process large-scale datasets
% - Example: The use of preference learning for virtual democracy
% - Example: Ranking from discrete comparisons
% - Example: Search

The history of studying choices to understand human behaviour dates from the 1920s.
% Describe work of Thurstone and Zermelo.
% In the psychometrics community.

The Thurstone model was independently rediscovered in the statistics community in the early 1950s by Bradley and Terry.
% Describe work of Bradley and Terry.

In the early 1960s, Marschak introduces Thurstone's work to the econometrics community.
% Describe work of Marschak, Luce (IIA), and McFadden (conditional logit)

Today, the availability of unprecedented computational power and large-scale datasets have enable new applications of discrete-choice models.
% Describe the work of Anson for preference learning in virtual democracy.
% Describe the work of Lucas for ranking.
% Describe the work of Daniyar for search.

The interested reader can find a more detailed introduction to random utility models in Train's and Henscher's books.
A fascinating, detailed history of the development of the discrete-choice models in econometrics is given by McFadden in his Nobel prize statement.
Another approach to introducing probabilistic models of choice is given in Maystre's thesis introduction.

\subsection{Random Utility Models}

We introduce discrete-choice models from a \emph{random utility} perspective.

\paragraph{Choice Set}
% - Definition and notation of a choice set.
% - Three characteristics: mutually exclusive, exhaustive, finite

\paragraph{Random Utility}
% Define a random utility and the error term.
% Explain that the main assumption to be made is on the error term.

\paragraph{Irrelevant and Independent Alternatives}

\paragraph{Probit Model}

\paragraph{Logit Model}
% Define the logit model, \textit{i.e.}, the Bradley-Terry model

\paragraph{Conditional Logit Model}
% Define the multinomial logit model by McFadden.

\paragraph{Item Response Theory}
% Describe Rash model and make a connection to random utility model.

\subsection{Parameter Estimation}
% Derivation of stochastic gradient descent algorithm for the Bradley-Terry model
% Define the likelihood for the Bradley-Terry model.
% Compute the gradient for one parameter.
% Show the update rule with some interpretation.
% Point to more efficient estimation procedures (MM, ChoiceRank, ...)
