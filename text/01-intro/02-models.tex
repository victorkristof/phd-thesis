%! TEX root = ../thesis.tex
\section{Probabilistic Models of Choice}
\label{in:sec:models}

\subsection{A Brief History of Choice Models}
% Why studying choices to study human behaviour is important and useful
% - Long literature from psychometrics and econometrics
% - New methods enabled by computer science to process large-scale datasets
% - Example: The use of preference learning for virtual democracy
% - Example: Ranking from discrete comparisons
% - Example: Search

% Describe work of Thurstone and Zermelo.
% In the psychometrics community.
The history of studying choices to understand human behaviour originates from the 1920s in the psychometrics community.
\citet{thurstone1927law} pioneered the ``law of comparative judgement'', which established the methodology of measuring the perception of physical \emph{stimuli}, \textit{e.g.}, the weight of different objects, from pairwise comparisons.
The same year, \citet{thurstone1927method} used this new approach to study people's perception of the seriousness of crimes, a judgement for which no physical scale exists.
Almost concurrently, \citet{zermelo1928berechnung} proposed a similar model to rank chess players from match outcomes\footnote{This approach is still used today by the World Chess Federation~\citep{elo1978rating}.}.
Thurstone's model was then independently rediscovered in the early 1950s in the statistics community by~\citet{bradley1952rank}.

% Describe work of Marschak, Luce (IIA), and McFadden (conditional logit)
Two decades later, \citet{marschak1959binary} introduced Thurstone's work to the econometrics community by interpreting the psychological stimuli of Thurstone's model as economic \emph{random utility}.
In parallel, \citet{luce1959individual} proposed its \emph{choice axiom} and the property of \emph{irrelevant and independent alternatives} (IIA), which states that the relative comparison of two alternatives is unaffected by additions and subtractions of other alternatives.
This property enabled Luce to extend the Bradley-Terry model to multi-way comparisons.
This extension was also proposed by~\citet{mcfadden1973conditional} to propose the \emph{conditional logit model}\footnote{This model is also often referred to as the \emph{multinomial logit model}.} from a random utility viewpoint.

% Research between 1970 until today
% Unification and new models to relax the IIA hypothesis and account for correlation between alternatives
The following decades were dedicated to the development of new discrete-choice models.
In particular, to relax the (rather restrictive) IIA hypothesis, researchers developed the \emph{nested logit model}~\citep{ben1973structure,williams1977formation,mcfadden1978modelling} and the \emph{mixed logit model}~\citep{boyd1980effect,cardell1980measuring,hensher2003mixed}.
Some efforts were also deployed to unify the different formulations of the Bradley-Terry model by~\citet{yellot1977relationship}.
% Ranking
In parallel, pairwise-comparison data started to be used for \emph{ranking} items~\citep{ford1957solution,buehlmann1963pairwise,wauthier2013efficient,negahban2017rank}.
% Inference
Research around the inference of discrete-choice models was also conducted for \emph{sampling and simulations}~\citep{manski1981alternative,cosslett1981efficient}, \emph{maximum likelihood estimation}~\citep{hastie1998classification,hunter2004mm,maystre2015fast,vojnovic2016parameter}, and \emph{Bayesian inference}~\citep{guiver2009bayesian,caron2012efficient,houlsby2012collaborative}.

Today, the availability of unprecedented computational power and large-scale datasets have enabled new applications of discrete-choice models.
% Describe the work of Lucas for ranking and recommendation \citet{ailon2010preference,ammar2015ranked}
% Describe the work of Daniyar for search.
% Describe the work of Anson for preference learning in virtual democracy.
% Describe its use in reinforcement learning.

In the next section, we introduce discrete-choice models from a \emph{random utility} perspective.
The interested reader can find a more detailed introduction to random utility models in the books of~\citet[Chapter~1]{train2009discrete} and~\citet[Chapter~3]{hensher2005applied}.
A detailed history of the development of discrete-choice models in econometrics is given by~\citet{mcfadden2001economic} in his Nobel Prize lecture.
An introduction to probabilistic models of choice from a statistics perspective is given by~\citet[Chapter~1]{maystre2018efficient}.

\subsection{Random Utility Models}

\paragraph{Choice Set}
% - Definition and notation of a choice set.
% - Three characteristics: mutually exclusive, exhaustive, finite

\paragraph{Random Utility}
% Define a random utility and the error term.
% Explain that the main assumption to be made is on the error term.

\paragraph{Irrelevant and Independent Alternatives}

\paragraph{Probit Model}

\paragraph{Logit Model}
% Define the logit model, \textit{i.e.}, the Bradley-Terry model

\paragraph{Conditional Logit Model}
% Define the multinomial logit model by McFadden.

\paragraph{Item Response Theory}
% Describe Rash model and make a connection to random utility model.

\subsection{Parameter Estimation}
% Derivation of stochastic gradient descent algorithm for the Bradley-Terry model
% Define the likelihood for the Bradley-Terry model.
% Compute the gradient for one parameter.
% Show the update rule with some interpretation.
% Point to more efficient estimation procedures (MM, ChoiceRank, ...)
