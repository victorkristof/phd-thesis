\section{Motivation}
\label{in:sec:motivation}

While accelerating scientific and engineering progress at unprecedented speed, the effect of computers and algorithms on society is mixed.
% Elections have been perturbed by the scandal of Cambridge Analytica.
% Misinformation, fake news, and conspiracy theories spread to millions of people within algorithmically recommended echo chambers on social networks.
% Personalized news feeds on social media hook people behind their smartphones (TODO: cite).
% Bitcoin mining emits as much CO2 as Denmark, and training large language models, such as GPT-3, emits as much CO2 as driving a car for 10 years (TODO: verify).

These global problems, stemming from the decisions and behaviors of people, can seem intractable and unpredictable, but a century of research in econometrics and psychometrics has taught us that people are more predictable than we think.
% We make choices as soon as we get up, for example by choosing between wearing a pull-over or a shirt and between drinking tea or coffee.
% The discrete-choice theory was developed to analyze and forecast decision-making processes.
% This earned Daniel McFadden his Nobel prize.
% DCM offers tools to understand people's preferences in a variety of settings.
% They have gained an increasing interest with increasing computational power and larger datasets.

With large enough datasets and machine-learning algorithms, we can design accurate models of human behavior.
% At work and at home, we spend countless of hours behind a computer screen and on the internet, where every click and mouse movement is recorded.
% With the development of the Internet-of-Things, social media platforms, and smartphones, we are generating XX Gbytes (TODO: find amount of data generated per day) per day.
% In parallel, the development of machine-learning algorithms and the rapid increase of computational power enable us to process this vast amount of data.

While recent machine-learning algorithms offer unprecedented predictive powers, they offer little insights into the problem itself, limiting an in-depth understanding of human behaviour.
% Very often, these algorithms are used as black boxes, \textit{i.e.}, as oracles that gobble a dataset and spits out predictions.
% Other advanced algorithms in natural language processing and computer vision have so many layers of transformations and parameters that researchers are only guessing what the model has learned.

In this thesis, we focus on designing probabilistic models of decision-making that are highly interpretable.
We draw from the literature on discrete-choice models, which we combine with latent-factor models, Bayesian statistics, and generalized linear models.
Our models are general enough to be applicable in various contexts, but we follow a data-driven approach to study concrete problems.
We answer the following research questions:
\begin{enumerate}[
		% leftmargin=1cm,
		% topsep=1.5pt,
		% parsep=1.5pt,
		% itemsep=0pt,
		label=\textbf{RQ\arabic*}
	]
	\item Who are the influential users and the controversial components in peer-production systems?
	\item What are the features of parliamentarians and laws that contribute to high probability of law amendments acceptance?
	\item What ideological patterns are contained in vote data and how predictable are election results?
	\item How people perceive the carbon footprint of their actions?
	\item How to learn pairwise-comparison models of time-dependent data?
\end{enumerate}
