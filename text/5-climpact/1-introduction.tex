%! TEX root = ../thesis.tex
\section{Introduction}
\label{clm:sec:introduction}
To put the focus on actions that have high potential for emission reduction, we must first understand whether people have an accurate perception of the carbon footprint of these actions.
If they do not, their efforts might be wasted.
As an example, recent work by \citet{wynes2017climate} shows that Canadian high-school textbooks encourage daily actions that yield negligible emission reduction.
Actions with a higher potential of emission reduction are poorly documented.
In this work, we model how people perceive the carbon footprint of their actions, which could guide educators and policy-makers.

In their daily life, consumers repeatedly face multiple options with varying environmental effects.
Except for a handful of experts, no one is able to estimate the absolute quantity of \COtwo\ emitted by their actions of say, flying from Paris to London.
Most people, however, are aware that taking the train for the same trip would release less \COtwo.
Hence, in the spirit of \citet{thurstone1927method} and \citet{salganik2015wiki} (among many others), we posit that the perception of a population can be probed by simple pairwise comparisons.
By doing so, we shift the complexity from the probing system to the model: Instead of asking difficult questions about each action and simply averaging the answers, we ask simple questions in the form of comparisons and design a non-trivial model to estimate the perception.
\textit{In fine}, human behaviour boils down to making choices: For example, we choose between eating local food and eating imported food; we do not choose between eating or not eating.
Our awareness of relative emissions between actions (of the same purpose) is often sufficient to improve our carbon footprint.

Our contributions are as follows.
First, we cast the problem of inferring a population's global perception from pairwise comparisons as a linear regression.
Second, we adapt a well-known active-learning method to maximize the information gained from each comparison.
We describe the model and the active-learning algorithm in Section~\ref{clm:sec:models}.
We design an interactive platform to collect real data for an experiment on our university campus, and we show early results in Section~\ref{clm:sec:results}.
Our approach could help climate scientists, sociologists, journalists, governments, and individuals improve climate communication and enhance climate mitigation.
