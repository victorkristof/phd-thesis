\documentclass{article}

% if you need to pass options to natbib, use, e.g.:
\PassOptionsToPackage{numbers, compress}{natbib}
% before loading neurips_2019

% ready for submission
\usepackage[final]{neurips_2019}

% to compile a preprint version, e.g., for submission to arXiv, add add the
% [preprint] option:
%     \usepackage[preprint]{neurips_2019}

% to compile a camera-ready version, add the [final] option, e.g.:
% \usepackage[final]{neurips_2019}

% to avoid loading the natbib package, add option nonatbib:
%     \usepackage[nonatbib]{neurips_2019}

\usepackage[utf8]{inputenc} % allow utf-8 input
\usepackage[T1]{fontenc}    % use 8-bit T1 fonts
\usepackage{hyperref}       % hyperlinks
\usepackage{url}            % simple URL typesetting
\usepackage{booktabs}       % professional-quality tables
\usepackage{amsfonts}       % blackboard math symbols
\usepackage{nicefrac}       % compact symbols for 1/2, etc.
\usepackage{microtype}      % microtypography
% Better maths.
\usepackage{amsmath,amsthm,amssymb,bm,mathtools}
% If possible, it is preferable to directly include PDF images.
\usepackage{graphicx}
\graphicspath{{fig/}}

%% Macros.

% "Transpose" symbol.
\newcommand{\tr}{^\intercal}
% Email.
\newcommand{\email}[1]{\href{mailto:#1}{\nolinkurl{#1}}}
% Arg{sort, min, max}.
\DeclareMathOperator*{\argsort}{arg\,sort}
\DeclareMathOperator*{\argmax}{arg\,max}
\DeclareMathOperator*{\argmin}{arg\,min}
% Shortcut CO\textsubscript{2}.
\newcommand\COtwo{CO\textsubscript{2}}
% "Given" symbol.
% \newcommand\given{\:\vert\:}
\newcommand\given{\;|\;}
% Vectors in bold
\renewcommand{\vec}[1]{\boldsymbol{#1}}
% Enumerate with bold numbers.
\newenvironment{enumerateb}
{\begin{enumerate}\renewcommand\labelenumi{\textbf\theenumi .}}
		{\end{enumerate}}
% An action item for the list.
\newcommand{\actiontitle}[1]{\textbf{#1}}
\newcommand{\actionvalue}[1]{\textbf{Carbon footprint:} #1 kg\COtwo-equivalent.}

% Paper title.
\title{A User Study of Perceived Carbon Footprint}

% The \author macro works with any number of authors. There are two commands
% used to separate the names and addresses of multiple authors: \And and \AND.
%
% Using \And between authors leaves it to LaTeX to determine where to break the
% lines. Using \AND forces a line break at that point. So, if LaTeX puts 3 of 4
% authors names on the first line, and the last on the second line, try using
% \AND instead of \And before the third author name.

\author{%
	Victor Kristof\\
	EPFL
	% \texttt{victor.kristof@epfl.ch} \\
	% examples of more authors
	\And
	Valentin Quelquejay-Leclère \\
	EPFL
	% \texttt{lucas.maystre@spotify.com} \\
	\And
	Robin Zbinden \\
	EPFL
	% \texttt{lucas.maystre@spotify.com} \\
	\AND
	Lucas Maystre \\
	Spotify
	% \texttt{lucas.maystre@spotify.com} \\
	\And
	Matthias Grossglauser \\
	EPFL
	% \texttt{matthias.grossglauser@epfl.ch} \\
	\And
	Patrick Thiran \\
	EPFL
	% \texttt{patrick.thiran@epfl.ch} \\
}

\begin{document}

\maketitle

%! TEX root = ../thesis.tex

Similar to a peer-production system, a body of law is an example of a dynamic corpus of text documents that are jointly maintained by a group of editors who compete and collaborate in complex constellations.
Our goal is to develop predictive models for this process, thereby shedding light on the competitive dynamics of parliamentarians who make laws.
For this purpose, we curated a dataset of \numprint{450000} legislative edits introduced by European parliamentarians over the last ten years.
An \textit{edit} modifies the status quo of a law, and could be in competition with another edit if it modifies the same part of that law.
We propose a model for predicting the success of such edits, in the face of both the \textit{inertia} of the status quo and the \textit{competition} between overlapping edits.
The parameters of this model can be interpreted in terms of the influence of parliamentarians and of the controversy of laws.
%We show that parliamentarians in charge of a law under consultation by a legislative committee have influence on the success of edits.

%! TEX root = ../thesis.tex
\section{Introduction}%
\label{sec:introduction}

% General introduction.
The past decade has seen the emergence of several open-government initiatives for the increase of administration transparency through the publication of governmental data.
These data are of great interest to parties, companies, sub- and supra-government entities, researchers, and citizens.
In particular, the results of referenda and election ballots in municipalities, districts, states, and countries are valuable for understanding the structure and the dynamics of politics.

% Describe the interest by media, parties, interest groups, and people of having an early prediction.
In this paper, we address the problem of vote prediction when only partial results are available.
The ability to predict the outcome of votes both before and during ballot counting is relevant to political parties, interest groups, polling agencies, news outlets, government authorities, and interested citizens.
These predictions help uncover voting patterns, \textit{e.g.}, to identify swing regions, to understand voting behaviours, and to detect fraud.
Political parties and interest groups can enhance their campaigning efforts.
Polling agencies and news outlets can optimize their surveying efforts.
Authorities can monitor the smooth functioning of the voting process.

% Describe the main problem we are solving.
We focus on national vote predictions during the ballot counting, \textit{i.e.}, after all eligible voters have cast their ballots, as government officials start count the valid votes in each region.
We predict national results by using sequential regional results, and we seek to obtain accurate predictions as early as possible, \textit{i.e.}, with a minimum number of regional results.
Typically, less populated regions release their official counts earlier than more populated ones.
Regions where remote voting is allowed release their results earlier than regions where this is not allowed.
In some countries, for example in the U.S., some regions vote earlier than others by design.
We will show that our model is able to exploit the correlations between regions and between votes to obtain accurate early predictions.

% Explain that we focus on Switzerland and why.
Switzerland offers a fascinating laboratory for vote prediction due to its direct-democracy system.
Swiss citizens are called to vote four times a year on referenda and popular initiatives \citep{confederation2019democracy, confederation2019popular}.
As a result, the amount and frequency of voting data produced in Switzerland remains unmatched by any other country.
We take Switzerland as an example to develop our methodology but, as shown in Section~\ref{sec:experiments}, our algorithm can be applied to other countries and in other settings.

% Overview of our model for vote outcome.
In Section~\ref{sec:methodology}, we propose an algorithm to predict national vote outcomes from a sequence of regional vote results.
Our model has two components:
First, it learns the correlations between regions and between votes from historical data by using singular value decomposition (SVD).
Second, after observing at least one regional result for a new vote, it uses the SVD as input features to a generalized linear model (GLM) to predict the unobserved regional results.
The national outcome is then easily obtained by weighted aggregation of the predicted and the observed regional results.
The SVD, computed only once on the historical data, is inexpensive in terms of complexity and enables interpretation.
By using different likelihoods in the GLM, we gain flexibility in predicting binary outcomes (for votes) or categorical outcomes (for elections).

% Our model can be generalized to elections and mention US and Germany as an example.
For Swiss votes, where people must answer "Yes" or "No" on each ballot, we show that a Gaussian and a Bernoulli likelihood provide the best performance.
We also explore what the SVD offers in terms of interpretation of voting patterns.
Furthermore, we show that we can predict the outcome of the popular vote of a U.S.\ presidential election by casting this problem as a binary choice between two candidates.
We predict the outcome of parliamentary elections in Germany, where people must choose between five political parties, using a categorical likelihood.
We describe our experiments on state-level and district-level results in Section~\ref{sec:experiments}.

% Mention Predikon.ch.
We also deploy a Web platform available to the general public to provide vote prediction for Switzerland.
Using an API developed by the Swiss government, we are able to make real-time predictions during the official counting with partial regional results.
We also provide a data-visualization tool to explore voting patterns and to understand how our model makes predictions.
We describe our platform in Section~\ref{sec:depsys}.

% Our contributions.
In summary, our contributions are as follows:
We propose an efficient, flexible, and accurate algorithm for predicting the national outcome of a referendum or an election from early regional results.
We curate a new dataset of sequential vote results in Switzerland, covering \numprint{330} votes and \numprint{2196} regions between 1981 and 2020.
We deploy an interactive Web platform to display real-time vote predictions in Switzerland, together with tools to explore and visualize our dataset.
The data and the code are available on \href{https://www.github.com/indy-lab/submatrix-factorization}{github.com/indy-lab/submatrix-factorization} and the Web platform is available on \href{http://www.predikon.ch}{www.predikon.ch}.

%! TEX root = ../thesis.tex
\section{Model}
\label{kks:sec:model}

% Introduction to the section.
In this section, we formally introduce our probabilistic model.
For clarity, we take a clean-slate approach and develop the model from scratch.
We discuss in more detail how it relates to prior work in Section~\ref{kks:sec:relwork}.

% Features have latent score process that follows a GP
The basic building blocks of our model are \emph{features}\footnote{%
	In the simplest case, there is a one-to-one mapping between competitors (e.g., teams) and features, but decoupling them offers increased modeling power.}.
Let $M$ be the number of features; each feature $m \in [M]$ is characterized by a latent, continuous-time Gaussian process
\begin{align}
	\label{kks:eq:score}
	s_m(t) \sim \GP[0, k_m(t, t')].
\end{align}
We call $s_m(t)$ the \emph{score process} of $m$, or simply its \emph{score}.
The \emph{covariance function} of the process, $k_m(t, t') \define \Exp{s_m(t) s_m(t')}$, is used to encode time dynamics.
A brief introduction to Gaussian processes as well as a discussion of useful covariance functions is given in Section~\ref{kks:sec:covariances}.
The $M$ scores $s_1(t), \dots, s_M(t)$ are assumed to be (a priori) jointly independent, and we collect them into the \emph{score vector}
\begin{align*}
	\bm{s}(t) = \begin{bmatrix}s_1(t) & \cdots & s_M(t) \end{bmatrix}\Tr.
\end{align*}

% Competitors are sparse linear combination of M features
For a given match, each opponent $i$ is described by a sparse linear combination of the features, with coefficients $\bm{x}_i \in \mathbf{R}^M$.
That is, the score of an opponent $i$ at time $t^*$ is given by
\begin{align}
	\label{kks:eq:compscore}
	s_i \define \bm{x}_i\Tr \bm{s}(t^*).
\end{align}
In the case of a one-to-one mapping between competitors and features, $\bm{x}_i$ is simply the one-hot encoding of opponent $i$.
More complex setups are possible: For example, in the case of team sports and if the player lineup is available for each match, it could also be used to encode the players taking part in the match~\citep{maystre2016player}.
Note that $\bm{x}_i$ can also depend contextually on the match.
For instance, it can be used to encode the fact that a team plays at home~\citep{agresti2012categorical}.

% Observations are parametrized by score difference.
Each observation consists of a tuple $(\bm{x}_i, \bm{x}_j, t^*, y)$, where $\bm{x}_i, \bm{x}_j$ are the opponents' feature vectors, $t^* \in \mathbf{R}$ is the time, and $y \in \mathcal{Y}$ is the match outcome.
We posit that this outcome is a random variable that depends on the opponents through their latent score difference:
\begin{align*}
	y \mid \bm{x}_i, \bm{x}_j, t^* \sim p( y \mid s_i - s_j ),
\end{align*}
where $p$ is a known probability density (or mass) function and $s_i, s_j$ are given by~\eqref{kks:eq:compscore}.
The idea of modeling outcome probabilities through score differences dates back to~\citet{thurstone1927law} and~\citet{zermelo1928berechnung}.
The likelihood $p$ is chosen such that positive values of $s_i - s_j$ lead to successful outcomes for opponent $i$ and vice-versa.

% Graphical representation.
A graphical representation of the model is provided in Figure~\ref{kks:fig:pgms}.
For perspective, we also include the representation of a static model, such as that of~\citet{thurstone1927law}.
Our model can be interpreted as ``conditionally parametric'': conditioned on a particular time, it falls back to a (static) pairwise-comparison model parametrized by real-valued scores.

\begin{figure}
	\centering
	\subcaptionbox{
		Static model
	}[3cm]{
		\begin{tikzpicture}

% Nodes.
\node[latent] (s)  {$s_m$};
\node[obs, below=of s] (y) {$y_n$};
\node[const, left=0.3cm of y, yshift=0.4cm] (x) {$\bm{x}_n$};

% Edges.
\edge {s} {y};
\edge[-] {x} {y};

% Plates.
\plate {scores} {(s)} {$M$}
\plate {observations} {(y)(x)} {$N$}
\end{tikzpicture}

	}
	% \hfill
	\subcaptionbox{
		Our dynamic model \label{kks:fig:model}
	}{
		\begin{tikzpicture}

% Nodes.
\node[latent] (s1)  {$s_{m1}$};
\node[latent, right=1cm of s1] (s2)  {$s_{m2}$};
\node[latent, right=1.5cm of s2] (sn)  {$s_{mN}$};
\node[const, above=0.5cm of s1] (t1) {$t_1$};
\node[const, above=0.5cm of s2] (t2) {$t_2$};
\node[const, above=0.5cm of sn] (tn) {$t_N$};
\node[obs, below=of s1] (y1) {$y_1$};
\node[obs, below=of s2] (y2) {$y_2$};
\node[obs, below=of sn] (yn) {$y_N$};
\node[const, left=0.2cm of y1, yshift=0.4cm] (x1) {$\bm{x}_1$};
\node[const, left=0.3cm of y2, yshift=0.4cm] (x2) {$\bm{x}_2$};
\node[const, left=0.3cm of yn, yshift=0.4cm] (xn) {$\bm{x}_N$};

% Edges.
\edge[-] {t1} {s1};
\edge[-] {t2} {s2};
\edge[-] {tn} {sn};
\edge[-] {x1} {y1};
\edge[-] {x2} {y2};
\edge[-] {xn} {yn};
\edge {s1} {y1};
\edge {s2} {y2};
\edge {sn} {yn};

\path (t2) -- node[auto=false]{\ldots} (tn);
\path (y2) -- node[auto=false]{\ldots} (yn);
\draw[line width=2pt] (s1) -- (s2);
\draw[line width=2pt] (s2) -- node[fill=white] {\ldots} (sn);

% Plates.
\plate {scores} {(s1)(s2)(sn)} {$M$}
\end{tikzpicture}

	}
	\caption{
		Graphical representation of a static model (left) and of the dynamic model presented in this chapter (right).
		The observed variables are shaded.
		For conciseness, we let $\bm{x}_n \define \bm{x}_{n,i} - \bm{x}_{n,j}$.
		Right: the latent score variables are mutually dependent across time, as indicated by the thick line.}
	\label{kks:fig:pgms}
\end{figure}

\paragraph{Observation Models}
Choosing an appropriate likelihood function $p(y \mid s_i - s_j)$ is an important modeling decision and depends on the information contained in the outcome $y$.
The most widely applicable likelihoods require only \emph{ordinal} observations, \textit{i.e.}, whether a match resulted in a win or a loss (or a tie, if applicable).
In some cases, we might additionally observe points (e.g., in association football, the number of goals scored by each team).
To make use of this extra information, we can model
\begin{enuminline}
	\item the number of points of opponent $i$ with a Poisson distribution whose rate is a function of $s_i - s_j$, or
	\item the points difference with a Gaussian distribution centered at $s_i - s_j$.
\end{enuminline}
A non-exhaustive list of likelihoods is given in Table~\ref{kks:tab:likelihoods}.

\begin{table}
	\centering
	\caption{
		Examples of observation likelihoods.
		The score difference is denoted by $d \define s_i - s_j$ and the Gaussian cumulative density function is denoted by $\Phi$.
	}
	\label{kks:tab:likelihoods}
	\centering
	\begin{tabular}{l lll}
		\toprule
		Name           & $\mathcal{Y}$        & $p(y \mid d)$                           & References                                      \\
		\midrule
		Probit         & $\{\pm 1 \}$         & $\Phi(yd)$                              & \citep{thurstone1927law, herbrich2006trueskill} \\
		Logit          & $\{\pm 1 \}$         & $[1 + \exp(-yd)]^{-1}$                  & \citep{zermelo1928berechnung, bradley1952rank}  \\
		Ordinal probit & $\{\pm 1, 0 \}$      & $\Phi(yd - \alpha), \ldots$             & \citep{glenn1960ties}                           \\
		Poisson-exp    & $\mathbf{N}_{\ge 0}$ & $\exp(yd - e^d) / y!$                   & \citep{maher1982modelling}                      \\
		Gaussian       & $\mathbf{R}$         & $\propto \exp[(y - d)^2 / (2\sigma^2)]$ & \citep{guo2012score}                            \\
		\bottomrule
	\end{tabular}
\end{table}


%%%%%%%%%%%%%%%%%%%%%%%%%%%%%%%%%
\subsection{Covariance Functions}
\label{kks:sec:covariances}

% Quick recap on Gaussian processes.
A Gaussian process $s(t) \sim \GP[0, k(t, t')]$ can be thought of as an infinite collection of random variables indexed by time, such that the joint distribution of any finite vector of $N$ samples $\bm{s} = [s(t_1) \cdots s(t_N)]$ is given by $\bm{s} \sim \DNorm{\bm{0}, \bm{K}}$, where $\bm{K} = [k(t_i, t_j)]$.
That is, $\bm{s}$ is jointly Gaussian with mean $\bm{0}$ and covariance matrix $\bm{K}$.
We refer the reader to~\citet{rasmussen2006gaussian} for an excellent introduction to Gaussian processes.

% Dynamics through covariance functions.
Hence, by specifying the covariance function appropriately, we can express prior expectations about the time dynamics of a feature's score, such as smooth or non-smooth variations at different timescales, regression to the mean, discontinuities, linear trends and more.
Here, we describe a few functions that we find useful in the context of modeling temporal variations.
Figure~\ref{kks:fig:covariances} illustrates these functions through random realizations of the corresponding Gaussian processes.

\begin{figure}
	\centering
	\includegraphics[width=\textwidth]{kks-covariances}
	\caption{Random realizations of a zero-mean Gaussian process with six different covariance functions.}
	\label{kks:fig:covariances}
\end{figure}

\begin{description}
	\item[Constant] This covariance captures processes that remain constant over time.
	      It is useful in composite covariances to model a constant offset (i.e., a mean score value).

	\item[Piecewise Constant]
	      Given a partition of $\mathbf{R}$ into disjoint intervals, this covariance is constant inside a partition and zero between partitions.
	      It can, for instance, capture discontinuities across seasons in professional sports leagues.

	\item[Wiener] This covariance reflects Brownian motion dynamics (c.f. Section~\ref{kks:sec:relwork}).
	      It is non-stationary: the corresponding process drifts away from $0$ as $t$ grows.

	\item[Matérn] This family of stationary covariance functions can represent smooth and non-smooth variations at various timescales.
	      It is parametrized by a variance, a characteristic timescale and a smoothness parameter $\nu$.
	      When $\nu = 1/2$, it corresponds to a mean-reverting version of Brownian motion.

	\item[Linear] This covariance captures linear dynamics.
\end{description}

% Composite covariance functions.
Finally, note that composite functions can be created by adding or multiplying covariance functions together.
For example, let $k_a$ and $k_b$ be constant and Matérn covariance functions, respectively.
Then, the composite covariance $k(t, t') \define k_a(t, t') + k_b(t, t')$ captures dynamics that fluctuate around a (non-zero) mean value.
\citet[Section 2.3]{duvenaud2014automatic} provides a good introduction to building expressive covariance functions by composing simple ones.

%! TEX root = ../thesis.tex
\section{Results}
\label{sec:results}

Starting with no information at all, we arbitrarily set the prior noise $\sigma_n^{2} = 1$ and the prior covariance matrix to a spherical covariance $\vec{\Sigma}_p = \sigma_p^{2} \vec{I}$, with $\sigma_p^{2} = 10$.
Our results are qualitatively robust to a large range of values for $\sigma_p^{2}$.
In order to compare the perceived carbon footprint $\exp \overline{\vec{w}}$ with its true value $\exp \vec{v}$, we set the prior mean to $\vec{\mu} = c \vec{1}$, where $ c = \frac{1}{M}\sum_{i=1}^M v_i$ is the mean of the (log-)true values.
This guarantees that the perceived carbon footprint estimated from the model parameters have the same scale as the true values.

We compile a set $\mathcal{A}$ of $M=18$ individual actions about transportation, food, and household (the full list of actions is provided in Appendix~\ref{app:actions}).
We deploy an online quiz\footnote{Accessible at \url{http://www.climpact.ch}} to collect pairwise comparisons of actions from real users on a university campus.
We collect $N=2183$ triplets from 176 users, mostly students between 16 and 25 years old.
We show in Figure~\ref{fig:perception} the true carbon footprint, together with the global perception of the population, i.e., the values $\exp \overline{w}_i$ for each action $i \in \mathcal{A}$.

\begin{figure}
	\centering
	\includegraphics{clm-perception.pdf}
	\caption{Global perceived carbon footprint of 18 actions in kg\COtwo-equivalent and their true values (log scale).
		The list of actions is provided in Appendix~\ref{app:actions}.}%
	\label{fig:perception}
\end{figure}

The users in our population have a globally accurate perception.
Among the actions showing the most discrepancy, the carbon footprint of short-haul flights is \textit{overestimated}~(Action 11), whereas the carbon footprint of long-haul flights~(16) is highly \textit{underestimated}~(the scale is logarithmic).
Similarly, the carbon footprint of first-class flights~(18) is also \textit{underestimated}.
The users tend to \textit{overestimate} the carbon footprint of more ecological transports, such as the train, the bus, and car-sharing~(1, 4, and 6).
The users have an accurate perception of actions related to diet~(8, 14, and 15) and of actions related to domestic lighting~(3 and 10).
They \textit{overestimate}, however, the carbon footprint of a dryer~(2).
Finally, they highly \textit{underestimate} the carbon footprint of oil heating~(17).
Switzerland, where the users live, is one of the European countries whose consumption of oil for heating houses is the highest.
There is, therefore, a high potential for raising awareness around this issue.

\input{4-conclusion}

\newpage
\bibliographystyle{abbrvnat}
\bibliography{climpact}

\appendix
\include{5-appendix}

\end{document}
