%! TEX root = ../thesis.tex
\section{Probabilistic Models of Choice}
\label{in:sec:models}

\subsection{A Brief History}
% Why studying choices to study human behaviour is important and useful
% - Long literature from psychometrics and econometrics
% - New methods enabled by computer science to process large-scale datasets
% - Example: The use of preference learning for virtual democracy
% - Example: Ranking from discrete comparisons
% - Example: Search

% Describe work of Thurstone and Zermelo.
% In the psychometrics community.
The history of studying choices to understand human behaviour originates from the 1920s in the psychometrics community.
\citet{thurstone1927law} pioneered the ``law of comparative judgement'', which established the methodology of measuring the perception of physical \emph{stimuli} (\textit{e.g.}, the weight of different objects) from pairwise comparisons.
The same year, \citet{thurstone1927method} used this new approach, today known as the \emph{probit model}, to study people's perception of the seriousness of crimes, a judgement for which no physical scale exists.
Almost concurrently, \citet{zermelo1928berechnung} proposed a similar model, known as the \emph{logit model}, to rank chess players from match outcomes\footnote{This approach is still used today by the World Chess Federation~\citep{elo1978rating}.}.
Zermelo's model was then independently rediscovered in the early 1950s in the statistics community by~\citet{bradley1952rank}.

% Describe work of Marschak, Luce (IIA), and McFadden (conditional logit)
In the late 1950s, \citet{marschak1959binary} introduced Thurstone's work to the econometrics community by interpreting the psychological stimuli of Thurstone's model as economic \emph{random utility}.
In parallel, \citet{luce1959individual} proposed its \emph{choice axiom} and the property of \emph{independence of irrelevant alternatives} (IIA), which states that the relative comparison of two alternatives is unaffected by additions and subtractions of other alternatives, \textit{i.e.}, the alternatives are uncorrelated.
This property enabled Luce to extend the logit model to multi-way comparisons.
This extension was also proposed by~\citet{mcfadden1973conditional} to introduce the \emph{multinomial logit model}\footnote{This model was first introduced as the \emph{conditional logit model}.} from a random utility viewpoint.

% Research between 1970 until today
% Unification and new models to relax the IIA hypothesis and account for correlation between alternatives
The following decades were dedicated to extend discrete-choice models.
In particular, to relax the (rather restrictive) IIA hypothesis, researchers developed the \emph{nested logit model}~\citep{ben1973structure,williams1977formation,mcfadden1978modelling} and the \emph{mixed logit model}~\citep{boyd1980effect,cardell1980measuring,hensher2003mixed}.
Some efforts were also deployed by~\citet{yellot1977relationship} to unify the different formulations of the logit model.
% Ranking
In parallel, pairwise-comparison data started to be used for \emph{ranking}~\citep{ford1957solution,buehlmann1963pairwise,plackett1975analysis,wauthier2013efficient,negahban2017rank}, a model often referred to as the \emph{Placket-Luce model}.
% Inference
Research around the inference of discrete-choice models was also conducted for \emph{sampling and simulations}~\citep{manski1981alternative,cosslett1981efficient}, \emph{maximum likelihood estimation}~\citep{hastie1998classification,hunter2004mm,maystre2015fast,vojnovic2016parameter}, and \emph{Bayesian inference}~\citep{guiver2009bayesian,caron2012efficient,houlsby2012collaborative}.

Today, the availability of unprecedented computational power and large-scale datasets has enabled new applications of discrete-choice models.
% Describe its use in reinforcement learning.
In reinforcement learning, \citet{sadigh2017active} and~\citet{christiano2017deep} proposes to use pairwise-comparison models to incorporate feedback from human supervisors into the reward function.
% Describe the work of Lucas for ranking and recommendation \citet{ammar2015ranked}
\citet{ammar2015ranked} suggests to use these models for making personalized recommendations.
% Describe the work of Daniyar for search.
\citet{chumbalov2020scalable} proposes a search algorithm to navigate large-scale databases of complex items (\textit{e.g.}, images) using pairwise comparisons.
% Describe the work of Anson for preference learning in virtual democracy and AllOurIdeas.
\citet{lee2019webuildai} uses the Plackett-Luce model in a \emph{virtual democracy} setting to learn people's preferences for making algorithmic policies.
\citet{noothigattu2018voting} also uses this model to train autonomous vehicles to make ethical decisions.
Finally, \citet{salganik2015wiki} implemented the probit model into the online platform All Our Ideas\footnote{\href{http://www.allourideas.org/planyc_example?guides=true}{http://www.allourideas.org}} to help the New York City Mayor’s Office of Long-Term Planning and Sustainability understand New Yorkers' preferences for developing the city sustainably.

A history of the development of discrete-choice models in econometrics is given by~\citet{mcfadden2001economic} in his Nobel Prize lecture.
The interested reader will find more details about random utility models in the books of~\citet[Chapter~1]{train2009discrete} and~\citet[Chapter~3]{hensher2005applied}.
An introduction to probabilistic models of choice from a statistics perspective is given by~\citet[Chapter~1]{maystre2018efficient}.
In the next section, we introduce discrete-choice models from a \emph{random utility} perspective.

\subsection{Random Utility Models}

\subsubsection{Choice Set}
% - Definition and notation of a choice set.
% - Three characteristics: mutually exclusive, exhaustive, finite
Discrete-choice models enable the analysis of people's preferences that drive their choice.
Making a choice is a decision-making process.
When facing a set of (at least two) alternatives, a decision maker chooses one of the alternative over the others.
This set of alternatives is defined as the \emph{choice set}.

\begin{definition}[Choice Set]
  Given the set of all possible alternatives~$\mathcal{A}$, the \emph{choice set}~$\mathcal{C} \subseteq \mathcal{A}$ is the set of alternatives faced by a decision maker.
  It has the following three characteristics:
  \begin{enumerate}
    \item The alternatives are \emph{mutually exclusive}.
    \item The choice set $\mathcal{C}$ is \emph{exhaustive}.
    \item The number of alternatives is \emph{finite}.
  \end{enumerate}
\end{definition}

The exclusiveness of alternatives means that, when choosing alternative $i \in \mathcal{C}$, the other alternatives of $\mathcal{C}$ are implicitly left aside.
The exhaustiveness of the choice set means that the decision maker face all possible alternatives at decision time.
Finally, the number of alternatives must be finite:
The decision maker can count the alternatives.

The first two characteristics are not restrictive because it is always possible to add artificial alternatives.
For example, if $\mathcal{C} = \{i, j\}$, one can ensure exclusiveness by adding a third alternative $k = \text{"choose $i$ and $j$"}$.
This allows the decision maker to choose both alternatives at the same time.
Similarly, one can ensure exhaustiveness by adding another alternative $l = \text{"none of the alternatives"}$.
This allows the decision maker to choose none of the alternatives, hence making the choice set exhaustive.

The third characteristics is, however, restrictive.
A finite number of alternatives is actually the defining characteristics of discrete-choice models.
This contrasts with regression models, in which the target variable is continuous, and hence the number of alternatives is infinite.
The choice set can also vary for each choice faced by a decision maker.
For example, $\mathcal{C}_1 = \{ i, j \}$ and $\mathcal{C}_2 = \{ i, j, k \}$.
This contrasts with classification models, in which the choice set (\textit{i.e.}, the domain of the target variable) is identical for every observation.
For example, $\mathcal{C}_1 = \mathcal{C}_2 = \{ \text{``spam''}, \text{``ham''} \}$.

\subsubsection{Random Utility}
% Define a random utility and the error term.
% Explain that the main assumption to be made is on the error term.

Without loss of generality, we introduce the random utility models from an econometrics viewpoint by analyzing the behavior of decision makers.
These models can, obviously, be used to model other processes.
In particular, and as we will see in this thesis, these models can model \emph{implicit} choices.
For example, in the context of collective-sport matches, team~$A$ wins over team~$B$, \textit{i.e.}, team~$A$ is implicitly chosen over team~$B$ (\textit{e.g.}, because it played better or was more lucky).

In econometrics, we posit that a decision maker chooses the alternative which maximizes its personal gain.
Let~$\bm{x}_i \in \mathbf{R}^M$ be a vector of $M$ observable features that might influence a decision maker's decision of choosing alternative~$i \in \mathcal{C}$, and let $\bm{w}\in \mathbf{R}^M$ be the associated $M$-dimensional parameter vector.
Let also~$\epsilon_i$ be some random noise that captures all unobservable features affecting the decision maker's choice, and whose probability distribution is to be defined.
The \emph{random utility}~$U_i$ of alternative~$i$ is given by
\begin{equation}
  U_i = \bm{x}_i\Tr \bm{w} + \epsilon_i,
\end{equation}
and the decision maker chooses alternative~$i$ if $U_i > U_j$, for all $j \neq i$, $i,j \in \mathcal{C}$.
Hence, the probability that a decision maker chooses~$i$ over~$j$ is
\begin{equation}
  \Prob{i \succ j} = \Prob{U_i > U_j} = \Prob{ \bm{x}_i\Tr \bm{w} + \epsilon_i > \bm{x}_j\Tr \bm{w} + \epsilon_j} = \Prob{\epsilon_i - \epsilon_j > \bm{x}_i\Tr \bm{w} - \bm{x}_j\Tr \bm{w}}.
\end{equation}
The probability $\Prob{i \succ j}$ is called \emph{choice probability}.
The notation ``$i \succ j$'' reads as ``alternative~$i$ is chosen over alternative~$j$'' or, equivalently, as ``$i$ wins over $j$''.

The characterization of a discrete-choice model hence depends on the researcher's hypotheses on the noise model, \textit{i.e.}, what probability distribution captures best the unobserved features.
Two popular choices of distribution, that we describe below, are (i) the Gaussian distribution, \textit{i.e.}, $\epsilon_i \sim \DNorm{0, 1}$, and (ii) the Gumbel distribution, \textit{i.e.}, $\epsilon_i \sim \DGumbel{0, 1}$.

\paragraph{Probit Model}

The probit model was first introduced by~\citet{thurstone1927law} in the context of psychometrics.
In this model, the noise model is independently and identically distributed with the Gaussian distribution, \textit{i.e.}, $\epsilon_i, \epsilon_j \sim \DNorm{0, 1}$.
As the difference of two Gaussian random variables is also Gaussian, \textit{i.e.}, $\epsilon_i - \epsilon_j \sim \DNorm{0, 1}$ in this special case, the choice probability for the probit model is
\begin{equation}
  \Prob{i \succ j} = \Prob{\epsilon_i - \epsilon_j > \bm{x}_i\Tr \bm{w} - \bm{x}_j\Tr \bm{w}} = \Phi \left( \bm{x}_i\Tr \bm{w} - \bm{x}_j\Tr \bm{w} \right),
\end{equation}
where $\Phi( \cdot )$ is the cumulative distribution function of the standard normal distribution.
When the random utility is parameterized by only one parameter, $U_i = \bm{x}_i\Tr \bm{w} + \epsilon_i = w_i + \epsilon_i$, and the model
\begin{equation}
  \Prob{i \succ j} = \Phi \left( w_i - w_j \right)
\end{equation}
is called the \emph{Thurstone model}.
In this simpler scenario, the $M$ parameters~$\bm{w} = [w_1 \cdots w_M]\Tr$ capture a \emph{score} for each of the $M$~alternative.
They can be interpreted as the \emph{perceived psychological stimuli} of the alternatives and enable us to rank them according to these stimuli.

\paragraph{Logit Model}
% Define the logit model, \textit{i.e.}, the Bradley-Terry model

The logit model was introduced by~\citet{zermelo1928berechnung} and rediscovered two decades later by~\citet{bradley1952rank}.
In this model, the random noise is assumed to be independently and identically distributed with a Gumbel distribution\footnote{This distribution is also called the (Type I) extreme value distribution.}, \textit{i.e.}, $\epsilon_i, \epsilon_j \sim \DGumbel{0, 1}$.
By using the property that the difference of two Gumbel random variables follows a logistic distribution, the choice probability for the logit model is
\begin{equation}
  \Prob{i \succ j} = \frac{\exp(\bm{x}_i\Tr \bm{w})}{\exp(\bm{x}_i\Tr \bm{w}) + \exp(\bm{x}_j\Tr \bm{w})} = \frac{1}{1 + \exp[-(\bm{x}_i\Tr \bm{w} - \bm{x}_j\Tr \bm{w})]}. %= \sigma(\bm{x}_i\Tr \bm{w} - \bm{x}_j\Tr \bm{w}).
\end{equation}
When the random utility is parameterized by only one parameter, the model
\begin{equation}
  \label{eq:bradley-terry}
  \Prob{i \succ j} = \frac{1}{1 + \exp[-(w_i - w_j)]}
\end{equation}
is called the \emph{Bradley-Terry model}.
In this scenario, the parameters~$\bm{w}$ can be interpreted as the intrinsic \emph{strength} of each alternative.

\paragraph{Rasch Model}
% Describe the Rash model and make a connection to random utility model.
While not categorized as a discrete-choice model, the Rasch model is closely related to the Bradley-Terry model.
We present it here because we combine the Rasch model with the multinomial logit model in Chapter~\ref{ch:lmp}.
The Rasch model was introduced by~\citet{rasch1993probabilistic} in the context of \emph{item response theory} to measure people's ability to answer tests and understand the traits explaining their performance.
It assumes that an individual~$u$ taking a test has an intrinsic strength~$s_u \in \mathbf{R}$, and that a question~$i$ in the test has an intrinsic difficulty~$d_i \in \mathbf{R}$.
The probability that individual~$u$ answers correctly to question~$i$ is
\begin{equation}
  \label{eq:rasch}
  \Prob{u \succ i} = \frac{1}{1 + \exp[-(s_u - d_i)]}.
\end{equation}
The relation with the Bradley-Terry model follows directly from comparing Equation~\ref{eq:bradley-terry} with Equation~\ref{eq:rasch}.

\paragraph{Multinomial Logit Model}
% Define the multinomial logit model by McFadden.

The multinomial logit model, also called conditional logit model, was first introduced by~\citet{luce1959individual} and later by~\citet{mcfadden1973conditional}.
In the probit and logit models, the decision maker faces a binary choice, \textit{i.e.}, the size of the choice set is $\Abs{\mathcal{C}} = 2$.
In the multinomial logit model, the decision maker faces multiple alternatives, and the choice set $\mathcal{C} = \{ i, j, \ldots k \}$ is larger than 2.
The random noise is also assumed to be independently and identically distributed with the Gumbel distribution, such that the choice probability is
\begin{equation}
  \label{eq:multinomiallogit}
  \Prob{i \succ \mathcal{C}} = \Prob{U_i > U_j, \ldots, U_i > U_k} = \frac{\exp(\bm{x}_i\Tr \bm{w})}{\sum_{j \in \mathcal{C}} \exp(\bm{x}_j\Tr \bm{w})}.
\end{equation}
The notation ``$i \succ \mathcal{C}$'' reads as ``alternative~$i$ is chosen among all alternatives in the choice set $\mathcal{C}$''.

\paragraph{Independence of Irrelevant Alternatives}

The property of \emph{independence of irrelevant alternatives} (IIA) was first formulated by~\citet{luce1959individual} as a lemma derived from its choice axiom.
The IIA states that, for any two alternatives~$i$ and~$k$, the ratio of the multinomial logit probabilities from Equation~\ref{eq:multinomiallogit} is independent of alternatives other than~$i$ and~$k$, \textit{i.e.},
\begin{equation}
  \frac{\Prob{i \succ \mathcal{C}}}{\Prob{k \succ \mathcal{C}}}
  = \frac{\exp(\bm{x}_i\Tr \bm{w}) / \sum_{j \in \mathcal{C}} \exp(\bm{x}_j\Tr \bm{w})}{\exp(\bm{x}_k\Tr \bm{w}) / \sum_{j \in \mathcal{C}} \exp(\bm{x}_j\Tr \bm{w})}
  = \exp(\bm{x}_i\Tr \bm{w} - \bm{x}_k\Tr \bm{w}).
\end{equation}
As this ratio depends only on alternatives~$i$ and $k$, adding or removing alternatives from the choice set~$\mathcal{C}$ will leave it unchanged.
This is a powerful result because it implies that not all alternatives are necessary in order to obtain an estimate of the associated parameters.
As a result, under the multinomial logit model (i) the computational cost of estimating the parameters of many alternatives can be reduced by sub-sampling the alternatives and (ii) if one is only interested in analyzing alternatives~$i,j \in \mathcal{A}$, the other alternatives~$k \in \mathcal{A} - \{i, j\}$ are irrelevant.
If the multinomial logit model exhibits this property, it is also because Luce proved that this model stems for the IIA hypothesis.

As shown by the blue-bus/red-bus paradox, however, the IIA assumption is restrictive.
Suppose that, among a population of commuters, the probability of taking the bus (which is blue) compared to commuting by car is $\Prob{\text{``blue bus''} \succ \text{``car''}} = 1/3$, and hence $\Prob{\text{``car''} \succ \text{``blue bus''}} = 2/3$.
The ratio between these two probabilities is equal to 2.
Suppose now that the city adds a new bus (which is red) to their fleet.
While this should not affect the probability of commuters to use their car, the multinomial logit model predicts that $\Prob{\text{``blue bus''} \succ \text{``car''}} = \Prob{\text{``red bus''} \succ \text{``car''}} = 1/4$ and $\Prob{\text{``car''} \succ \text{``blue/red bus''}} = 1/2$, so that the ratio is still equal to 2.
It should be expected, however, that the probability of using one's car is unaffected by this new bus, \textit{i.e.}, $\Prob{\text{``blue bus''} \succ \text{``car''}} = \Prob{\text{``red bus''} \succ \text{``car''}} = 1/6$ and $\Prob{\text{``car''} \succ \text{``blue/red bus''}} = 2/3$.
In this case, the ratio is equal to 4.

To circumvent this issue, new models have been proposed in the econometrics literature.
For example, the \emph{mixed logit model}, \emph{nested logit model}, and \emph{multinomial probit model} are all robust to the IIA hypothesis.
They provide a model of choice probabilities when alternatives are correlated between each other or over time.

\subsection{Parameter Estimation}
% Derivation of stochastic gradient descent algorithm for the Bradley-Terry model
% Define the likelihood for the Bradley-Terry model.
% Compute the gradient for one parameter.
% Show the update rule with some interpretation.
% Point to more efficient estimation procedures (MM, ChoiceRank, ...)
