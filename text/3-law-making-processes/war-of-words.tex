% For submission.
\documentclass[sigconf]{acmart}

\graphicspath{{fig/}}

% Already loaded in acmart:
% - microtype
% - fontenc
% - graphicx

\usepackage[utf8]{inputenc}
\usepackage{url}
\usepackage{amsmath,amssymb,bm,mathtools}
\usepackage{bbm}
\usepackage{subfig}
\usepackage{numprint}

% Email.
% \newcommand{\email}[1]{\href{mailto:#1}{\nolinkurl{#1}}}
% Arg{sort, min, max}.
\DeclareMathOperator*{\argsort}{arg\,sort}
\DeclareMathOperator*{\argmax}{arg\,max}
\DeclareMathOperator*{\argmin}{arg\,min}
% Shortcut CO\textsubscript{2}.
\newcommand\COtwo{CO\textsubscript{2}}
% "Given" symbol.
\newcommand\given{\:\vert\:}
% Vectors in bold
\renewcommand{\vec}[1]{\boldsymbol{#1}}
% "Transpose" symbol.
\newcommand{\tr}{^\intercal}

% Legislatures.
\renewcommand{\th}{\textsuperscript{th} }
\newcommand{\rd}{\textsuperscript{rd} }
\newcommand{\ep}[1]{EP#1}
% \newcommand{\ep8}{EP8}


\newcolumntype{L}[1]{>{\raggedright\let\newline\\\arraybackslash\hspace{0pt}}m{#1}}
\newcolumntype{C}[1]{>{\centering\let\newline\\\arraybackslash\hspace{0pt}}m{#1}}
\newcolumntype{R}[1]{>{\raggedleft\let\newline\\\arraybackslash\hspace{0pt}}m{#1}}

% Removes double spacing after end of sentence.
% See: http://practicaltypography.com/one-space-between-sentences.html.
\frenchspacing

%% Rights management information.  This information is sent to you
%% when you complete the rights form.  These commands have SAMPLE
%% values in them; it is your responsibility as an author to replace
%% the commands and values with those provided to you when you
%% complete the rights form.
\setcopyright{iw3c2w3}
\copyrightyear{2020}
\acmYear{2020}
\acmDOI{10.1145/3366423.3380041}

%% These commands are for a PROCEEDINGS abstract or paper.
\acmConference[WWW '20]{Proceedings of The Web Conference 2020}{April 20--24, 2020}{Taipei, Taiwan}
\acmBooktitle{Proceedings of The Web Conference 2020 (WWW '20), April 20--24, 2020, Taipei, Taiwan}
\acmPrice{}
\acmISBN{978-1-4503-7023-3/20/04}
%% Update ISBN for Proceedings or Companion, can be found on completed rightsreview form

\settopmatter{printacmref=true}

\begin{document}

% The "title" command has an optional parameter, allowing the author to define a "short title" to be used in page headers.
\title[The Competitive Dynamics of Legislative Processes]{War of Words: The Competitive Dynamics \texorpdfstring{\\of Legislative Processes}{}}

% The "author" command and its associated commands are used to define the authors and their affiliations.
% Of note is the shared affiliation of the first two authors, and the "authornote" and "authornotemark" commands
% used to denote shared contribution to the research.
\author{Victor Kristof}
\affiliation{%
	% \institution{Ecole polytechnique fédérale de Lausanne}
	\institution{EPFL}
}
% \email{victor.kristof@epfl.ch}

\author{Matthias Grossglauser}
\affiliation{%
	\institution{EPFL}
}
% \email{matthias.grossglauser@epfl.ch}

\author{Patrick Thiran}
\affiliation{%
	\institution{EPFL}
}
% \email{patrick.thiran@epfl.ch}

% By default, the full list of authors will be used in the page headers. Often, this list is too long, and will overlap
% other information printed in the page headers. This command allows the author to define a more concise list
% of authors' names for this purpose.
% \renewcommand{\shortauthors}{V. Kristof et al.}

\begin{abstract}
	%! TEX root = ../thesis.tex

Similar to a peer-production system, a body of law is an example of a dynamic corpus of text documents that are jointly maintained by a group of editors who compete and collaborate in complex constellations.
Our goal is to develop predictive models for this process, thereby shedding light on the competitive dynamics of parliamentarians who make laws.
For this purpose, we curated a dataset of \numprint{450000} legislative edits introduced by European parliamentarians over the last ten years.
An \textit{edit} modifies the status quo of a law, and could be in competition with another edit if it modifies the same part of that law.
We propose a model for predicting the success of such edits, in the face of both the \textit{inertia} of the status quo and the \textit{competition} between overlapping edits.
The parameters of this model can be interpreted in terms of the influence of parliamentarians and of the controversy of laws.
%We show that parliamentarians in charge of a law under consultation by a legislative committee have influence on the success of edits.

\end{abstract}

% The code below is generated by the tool at http://dl.acm.org/ccs.cfm.
% Please copy and paste the code instead of the example below.
% TODO: SHOULD BE FILLED LATER
% \begin{CCSXML}
%   <ccs2012>
%   <concept>
%   <concept_id>10002951.10003227.10003351</concept_id>
%   <concept_desc>Information systems~Data mining</concept_desc>
%   <concept_significance>300</concept_significance>
%   </concept>
%   <concept>
%   <concept_id>10002951.10003260.10003277</concept_id>
%   <concept_desc>Information systems~Web mining</concept_desc>
%   <concept_significance>300</concept_significance>
%   </concept>
%   <concept>
%   <concept_id>10010147.10010257</concept_id>
%   <concept_desc>Computing methodologies~Machine learning</concept_desc>
%   <concept_significance>300</concept_significance>
%   </concept>
%   </ccs2012>
% \end{CCSXML}

% \ccsdesc[300]{Information systems~Data mining}
% \ccsdesc[300]{Information systems~Web mining}
% \ccsdesc[300]{Computing methodologies~Machine learning}

% Keywords. The author(s) should pick words that accurately describe the work being
% presented. Separate the keywords with commas.
% \keywords{\textcolor{red}{discrete choice models, computational social science, human behaviour modelling, dataset, text mining, web mining}}

\maketitle

%! TEX root = ../thesis.tex
\section{Introduction}
\label{kks:sec:intro}

% General context & problem setting
In many competitive sports and games (such as tennis, basketball, chess and electronic sports), the most useful definition of a competitor's skill is the propensity of that competitor to win against an opponent.
It is often difficult to measure this skill \emph{explicitly}:
take basketball for example, a team's skill depends on the abilities of its players in terms of shooting accuracy, physical fitness, mental preparation, but also on the team's cohesion and coordination, on its strategy, on the enthusiasm of its fans, and a number of other intangible factors.
However, it is easy to observe this skill \emph{implicitly} through the outcomes of matches.

% Static models of pairwise comparisons.
In this setting, probabilistic models of pairwise-comparison outcomes provide an elegant and practical approach to quantifying skill and to predicting future match outcomes given past data.
These models, pioneered by~\citet{zermelo1928berechnung} in the context of chess (and by~\citet{thurstone1927law} in the context of psychophysics), have been studied for almost a century.
They posit that each competitor $i$ (i.e., a team or player) is characterized by a latent score $s_i \in \mathbf{R}$ and that the outcome probabilities of a match between $i$ and $j$ are a function of the difference $s_i - s_j$ between their scores.
By estimating the scores $\{ s_i \}$ from data, we obtain an interpretable proxy for skill that is predictive of future match outcomes.
If a competitor's skill is expected to remain stable over time, these models are very effective.
But what if it varies over time?

% Dynamic models.
A number of methods have been proposed to adapt comparison models to the case where scores change over time.
Perhaps the best known such method is the Elo rating system~\citep{elo1978rating}, used by the World Chess Federation for their official rankings.
In this case, the time dynamics are captured essentially as a by-product of the learning rule (c.f. Section~\ref{kks:sec:relwork}).
Other approaches attempt to model these dynamics explicitly~\citep[e.g.,][]{fahrmeir1994dynamic, glickman1999parameter, dangauthier2007trueskill, coulom2008whole}.
These methods greatly improve upon the static case when considering historical data, but they all assume the simplest model of time dynamics (that is, Brownian motion).
Hence, they fail to capture more nuanced patterns such as variations at different timescales, linear trends, regression to the mean, discontinuities, and more.

% KickScore: value proposition.
In this work, we propose a new model of pairwise-comparison outcomes with expressive time-dynamics: it generalizes and extends previous approaches.
We achieve this by treating the score of an opponent $i$ as a time-varying Gaussian process $s_i(t)$ that can be endowed with flexible priors~\citep{rasmussen2006gaussian}.
We also present an algorithm that, in spite of this increased flexibility, performs approximate Bayesian inference over the score processes in linear time in the number of observations so that our approach scales seamlessly to datasets with millions of observations.
This inference algorithm addresses several shortcomings of previous methods: it can be parallelized effortlessly and accommodates different variational objectives.
The highlights of our method are as follows.

\begin{description}
	\item[Flexible Dynamics]
	      As scores are modeled by continuous-time Gaussian processes, complex (yet interpretable) dynamics can be expressed by composing covariance functions.

	\item[Generality]
	      The score of an opponent for a given match is expressed as a (sparse) linear combination of features.
	      This enables, e.g., the representation of a home advantage or any other contextual effect.
	      Furthermore, the model encompasses a variety of observation likelihoods beyond win / lose, based, e.g., on the number of points a competitor scores.

	\item[Bayesian Inference]
	      Our inference algorithm returns a posterior \emph{distribution} over score processes.
	      This leads to better predictive performance and enables a principled way to learn the dynamics (and any other model hyperparameters) by optimizing the log-marginal likelihood of the data.

	\item[Ease of Intepretation]
	      By plotting the score processes $\{ s_i(t) \}$ over time, it is easy to visualize the probability of any comparison outcome under the model.
	      As the time dynamics are described through the composition of simple covariance functions, their interpretation is straightforward as well.
\end{description}

% Contributions & organization.
Concretely, our contributions are threefold.
First, we develop a probabilistic model of pairwise-comparison outcomes with flexible time-dynamics (Section~\ref{kks:sec:model}).
The model covers a wide range of use cases, as it enables
\begin{enuminline}
	\item opponents to be represented by a sparse linear combination of features, and
	\item observations to follow various likelihood functions.
\end{enuminline}
In fact, it unifies and extends a large body of prior work.
Second, we derive an efficient algorithm for approximate Bayesian inference (Section~\ref{kks:sec:inference}).
This algorithm adapts to two different variational objectives;
in conjunction with the ``reverse-KL'' objective, it provably converges to the optimal posterior approximation.
It can be parallelized easily, and the most computationally intensive step can be offloaded to optimized off-the-shelf numerical software.
Third, we apply our method on several sports datasets and show that it achieves state-of-the-art predictive performance (Section~\ref{kks:sec:eval}).
Our results highlight that different sports are best modeled with different time-dynamics.
We also demonstrate how domain-specific and contextual information can improve performance even further;
in particular, we show that our model outperforms competing ones even when there are strong intransitivities in the data.

% Visualization & impact.
In addition to prediction tasks, our model can also be used to generate compelling visualizations of the temporal evolution of skills.
All in all, we believe that our method will be useful to data-mining practitioners interested in understanding comparison time-series and in building predictive systems for games and sports.
%Objects represent players or teams.
%Each datum represents a match with two opponents, in which we observe a winner (e.g., in tennis) or possibly a tie (e.g., in chess) or points (e.g., goals in football).
%The objective is to estimate the score of players or teams over time, in such a way that score differences predict match outcomes accurately.

\paragraph{A Note on Extensions}
In this paper, we focus on \emph{pairwise} comparisons for conciseness.
However, the model and inference algorithm could be extended to multiway comparisons or partial rankings over small sets of opponents without any major conceptual change, similarly to~\citet{herbrich2006trueskill}.
Furthermore, and even though we develop our model in the context of sports, it is relevant to all applications of ranking from comparisons, e.g., to those where comparison outcomes reflect human preferences or opinions~\citep{thurstone1927law, mcfadden1973conditional, salganik2015wiki}.

%! TEX root = ../thesis.tex
\section{The European Legislative Process}
\label{sec:background}

\begin{figure}
	\newcommand{\imgscale}{0.35}
	\includegraphics[scale=\imgscale]{lmp-ols}
	\caption{
		Sketch of the \textit{ordinary legislative procedure}.
		(A) The Commission submits a legislative proposal to one of the Parliament committees.
		(B) The proposal is amended and (C) submitted to vote to the whole Parliament.
		(D) If it is rejected, the proposal is abandoned.
		(E) If it is accepted, it is transferred to the Council.
		(F) If the Council accepts the amended proposal, a new law is adopted.
		(G) If the Council amends it, it is sent back to the committee.
		(H) Other committees can optionally make amendments and (I) suggest them to the reporting committee.
	}
	\label{fig:ols}

\end{figure}

\subsection{Representative Democracies}

In representative democracies, citizens elect politicians to represent them in the various branches of the government.
The executive branch is in charge of executing and enforcing the laws.
Representatives of the executive branch can also propose new laws, but, to avoid a concentration of power, they cannot pass new legislation without the approval of the legislative branch.
The legislative branch, typically a parliament, represents both the people and the sub-governmental entities (such as states and municipalities).
Parliamentarians  can propose new legislation or amend propositions made by the executive branch.
Finally, the judicial branch balances the power of the executive branch and the legislative branch through its ability to decide whether the laws are constitutional.

Here, we focus on the European Union (EU).
The EU is a political and economic union of 28 countries called member states.
This union enables them to share their markets, to ease mobility across borders, to favor economic development, and to harmonize laws.
The EU covers an estimated population of 513 million, and up to 84\% of member states' national laws emanate from the EU \cite{miller2010much}.
Hence, EU laws have a significant impact on the life of many people.
European institutions make efforts to be transparent.
They make a lot of valuable data available online: parliamentary amendments, meetings by the commissioners with civil society, and a transparency register to monitor interest groups.
%These efforts give access to large datasets of great value.

The EU political system is broadly similar to that of a regular state.
The 751 parliament representatives (MEPs, for Member of the European Parliament) are elected every five years by universal suffrage.
The executive branch is called the \textit{European Commission}.
The legislative branch consists of the \textit{European Parliament} and of the \textit{Council of Ministers}.
The Parliament is divided into 20 committees, comprising sub-sets of MEPs and specialized in some particular policy area (such as fisheries, judiciary affairs, transportation, and trade).
Each MEP is a member of at least one committee.
The myriad of national parties aggregate into a small number of political groups.

\subsection{The Ordinary Legislative Procedure}

We now describe the EU legislative process in some detail, leading up to our modeling assumptions.
Under the Treaty of Lisbon \cite{eu2007lisbon}, which marks the beginning of the 7\th legislature in 2009, the Parliament's powers were increased.
The Parliament became central in the process through which new laws are created.
This process can take the form of various procedures, the main one being the \textit{ordinary legislative procedure} (OLP)~\cite{europarl2018ordinary}.
Through the OLP, the Commission initiates a legislative proposal, and the Parliament must adopt it in order for the proposal to become a law.
Other procedures exist, where the Parliament is not necessarily involved.
Since 2009, the Parliament has dealt with 90\% of all new laws via the OLP.
In this regard, we focus on the dynamics of the legislative process in the Parliament.
A sketch of the OLP is illustrated in Figure~\ref{fig:ols} and described in the next paragraphs.

To create a new law, (A) the Commission drafts a legislative \textit{proposal} and transfers it to the corresponding committee of the Parliament.
For instance, if the proposal introduces regulations on greenhouse-gas emissions, it is transferred to the Environment Committee.
The committee appoints a \textit{rapporteur} to lead the debate.
The role of the committee is to write a \textit{report} in the form of \textit{amendments} to the proposal, i.e., insertions in or deletions of parts of the proposal.
The rapporteur first seeks external expertise to draft a report.
Then, (B) other MEPs on the committee can in turn propose amendments to the proposal.
To constitute the final report to be submitted to the whole Parliament, each amendment by the rapporteur or by other MEPs is therefore voted on within the committee.
Once the committee finds a consensus, (C) they transfer the report to the whole Parliament.

In the plenary session, the Parliament holds a vote on the report.
(D) If rejected, the proposal is abandoned; (E) if accepted, the report, establishing the Parliament's position on the proposal, is transferred to the Council of Ministers.
The report is therefore an important document and the rapporteur has an important role to play.
The ministers (of the different EU countries) can accept the report, (F) in which case, the proposal is adopted with the Parliament's amendments and a new law is created; or they can make amendments, (G) in which case it is transferred back to the parliamentary committee.
At this stage, we say that a law has gone through the first \textit{reading}.

\begin{table*}
	\caption{Descriptive statistics of the \warofwords\ dataset.}
	\label{tab:dataset}
	\begin{tabular}{cccccccc}
		\toprule
		Legislature      & \# amendments     & \# edits          & \# conflicts     & \# MEPs        & \# dossiers     & \# data points    & \% accepted \\
		\midrule

		EP7 (2009--2014) & \numprint{108292} & \numprint{200407} & \numprint{40302} & \numprint{761} & \numprint{1089} & \numprint{126417} & 37.7\%      \\
		EP8 (2014--2019) & \numprint{128885} & \numprint{249086} & \numprint{56298} & \numprint{791} & \numprint{800}  & \numprint{141034} & 25.7\%      \\

		\bottomrule
	\end{tabular}
\end{table*}

Other committees can also independently decide to address an \textit{opinion} to the reporting committee.
For instance, the Transportation Committee might consider that it is also concerned by greenhouse-gas emissions and that it is entitled to give its opinion to the Environment Committee.
An opinion is similar to a report in that it contains amendments to the proposal.
It is created similarly to a report, i.e., (H) the opinion committee appoints a rapporteur to draft an opinion, and other MEPs can propose amendments.
(I) The opinion committee then transfers its opinion to the reporting committee.
An opinion differs from a report in that it is not voted by the whole Parliament.
Amendments to the opinions are, however, valuable to the reporting committee that often takes them into consideration.
We refer to reports and opinions as \textit{dossiers}.

This iterative process can be repeated up to three times (three readings).
The third reading, called conciliation, involves a negotiation between the Parliament and the Council.
During the 8\th legislature for example, 99\% of all laws were adopted after the first reading, i.e., after amendments made by both the Parliament and the Council, and 89\% were adopted directly after amendments by the Parliament, i.e., the Council accepted it without making amendments.

%! TEX root = ../thesis.tex
\section{Dataset}
\label{lmp:sec:dataset}

\subsection{Amendments \& Edits}

We collected a dataset of \numprint{237177} legislative amendments from the European Parliament website.\footnote{Data and code publicly available on \href{https://github.com/indy-lab/war-of-words}{https://github.com/indy-lab/war-of-words}.}
The dataset spans the 7\th legislature~(referred to as EP7), from 2009 to 2014, and the 8\th legislature~(EP8), from 2014 to 2019.
MEPs come from 28 different countries, and they belong to one of the 8 (EP7) or 9 (EP8) political groups.
An amendment consists of (i) one or several authors, (ii) the original text by the European Commission, and (iii) the amended text by the author(s).
We show an example of two amendments in their raw format in Figure~\ref{lmp:fig:amendment}.
The two amendments are proposed on Article 13 of a proposal about copyrights on the Internet.
Amendment 802 is proposed by three MEPs and consists of three edits:
(a) Inserting ``copyright'' (in green), (b) replacing ``by'' by ``uploaded by users of'' (in yellow), and (c) deleting the end of the title after ``providers'' (in red).
Amendment 803 is proposed by two other MEPs and consists of two edits:
(d) Replacing ``large'' by ``significant'' (in yellow) and (e) inserting ``copyright protected'' (in green).
There are two conflicts in this amendment:
Edit (c) of the first amendment is in conflict with Edit (d), and it is also in conflict with Edit (e).
All these edits are also implicitly in conflict with the original text proposed by the European Commission.
Out of these five edits, only Edit (d) was accepted.
All other edits were rejected, \textit{i.e.}, the status quo was voted and the text proposed by the Commission was maintained.
% We summarize our dataset in Table~\ref{lmp:tab:dataset} and we refer to it as the \warofwords\ dataset.
% In the next paragraphs, we describe the data that we extract from amendments and that we use for the subsequent analysis.
% We extract \textit{edits} and \textit{conflicts}, and we explain the labelization process.
% Technical details about data processing are given in Appendix~\ref{lmp:sec:dataproc}.

\begin{figure}
  \centering
	\newcommand{\imgscale}{\linewidth*3/4}
	{%
		\setlength{\fboxsep}{5.5pt}%
		\setlength{\fboxrule}{0.5pt}%
		\fbox{\includegraphics[width=\imgscale]{lmp-amendment-802-colors}}%
	}%
	\vfill
	\vspace{4pt}
	{%
		\setlength{\fboxsep}{5.5pt}%
		\setlength{\fboxrule}{0.5pt}%
		\fbox{\includegraphics[width=\imgscale]{lmp-amendment-803-colors}}%
	}%
	\caption{
		Example of two conflicting amendments in their raw format on the title of Article 13 of a proposal about copyrights on the Internet.
		(Top) Amendment 802 is proposed by three MEPs and consists of three edits.
		(Bottom) Amendment 803 is proposed by two other MEPs on the same text, and it consists of two edits.
		The last edit of Amendment 802 (deleting the end of the title) conflicts with both edits of Amendment 803.
		Only the first edit of Amendment 803 (replacing ``large'' by ``significant'') was accepted, and all other edits were rejected.
	}
	\label{lmp:fig:amendment}
\end{figure}

\paragraph{Edits}
MEPs propose amendments on a specific article of the legislation, and they can modify several parts within a single amendment.
As a result, we decompose the difference between the original and the amended text into one or several \emph{edits}, as defined below.
An edit is a sequence of words that are inserted or deleted or both.
We extract edits by computing the \emph{diff}, \textit{i.e.}, the difference between the words in two texts, between the original and the amended text of each amendment.
We normalize the texts by removing special characters and by putting the words in lower case.
We keep punctuation because the structure of sentences is important in legal texts.
We merge identical edits proposed by different MEPs, thus considering them as one edit proposed by all authors together.
This is in line with Rule 174 of the Rules of Procedure of the Parliament \cite{europarl2018rule174}.
We extract \numprint{200407} edits for EP7 and \numprint{249086} edits for EP8.
On average, there are 1.85 and 1.93 edits per amendment for EP7 and EP8, respectively.
There are also more dossiers in EP7 than in EP8, which means that there are proportionally more edits per dossier in EP8.

\paragraph{Conflicts}
There exists an inherent competition between the MEPs in the amending process, as amendments are vehicles of political ideas and interests.
We are therefore interested in the conflicts between edits.
We define a \emph{conflict} as a set of edits that overlap.
Edits overlap because they modify parts of the text at the same position.
We extract \numprint{40302} conflicts for EP7 and \numprint{56298} for EP8.
Adding the conflicts to isolated edits, we obtain a dataset of \numprint{126417} data points for EP7 and \numprint{141034} data points for EP8.

\paragraph{Labels}
The votes on each edit are not publicly available, and we need to infer their outcomes from the raw data.
Reports and opinions contain only the amendments accepted within the committees.
Draft reports, draft opinions, and other documents containing all proposed amendments are published separately.
Therefore, if the edits extracted from the latter documents  appear in the former documents, we label them as \emph{accepted}, \textit{i.e.}, the committee votes to include these edits in their report or opinion.
Otherwise, we label them as \emph{rejected}.
Out of the proposed edits, 37.7\% are accepted for EP7 and 25.7\% for EP8.

\paragraph{Timestamps}
The timeline of the legislative process described in Section~\ref{lmp:sec:background} varies from one dossier to another.
Depending on the dossier, MEPs can propose edits during a window of one to six months, after which all the edits related to that dossier are published together.
As a result, the actual, detailed chronology of the edits is unfortunately hidden, and we do not have access to the precise time the edits are proposed and when they are voted.
Furthermore, there is a delay between the time an edit is proposed and the time it is voted: recent edits might be voted \emph{before} older ones.
The timestamps associated with each edit are, therefore, noisy.

%This departs from reality, because we do not have access to the precise time the edits are proposed and when they are voted.
%We only know when a proposal has reached a reporting committee (respectively, an opinion committee), and when the report (respectively, the opinion) is transferred to the Parliament (respectively, the reporting committee).
%In Section~\ref{lmp:sec:cold-start}, we will explore ways of emulating an amending process that is closer to the actual process.
%As shown in Section~\ref{lmp:sec:results}, these simplistic assumptions enable us, nonetheless, to take advantage of the available data to gain insights into the EU legislative process.


In total, we collect \numprint{449493} edits from \numprint{237177} amendments in the European Parliament during the 7\th and the 8\th legislature periods\footnote{We do not collect data from EP9 (2019 -- 2023), as the amount of published data is too small at this time: The legislature period started in Fall 2019 and the Parliament's activities were slowed down due to the COVID-19 crisis in Spring 2020.} (referred to as EP7 and EP8), between 2009 and 2019 (each period lasts 5 years).
After gathering the edits according to the conflicts, we obtain \numprint{267451} conflicts for both EP7 and EP8, covering 1889 dossiers.
We summarize this dataset in Table~\ref{lmp:tab:dataset}.

\begin{table}
  \centering
	\caption{Descriptive statistics of our extended dataset.}
	\label{lmp:tab:dataset}
	\begin{tabular}{lrr}
		\toprule
    & EP7 (2009--2014) &EP8 (2014--2019) \\
		\midrule
  \# amendments & \numprint{108292} &\numprint{128885}  \\
 \# edits       & \numprint{200407} &\numprint{249086}  \\
 \# conflicts   & \numprint{126417} &\numprint{141034}  \\
 \# MEPs        & \numprint{761}    &\numprint{791}     \\
 \# dossiers    & \numprint{1089}   &\numprint{800}     \\
 \% accepted    & 37.7\%            &25.7\%             \\
 \% inserted    & 37.8\%            &37.9\%             \\
 \% deleted     & 22.0\%            &22.4\%             \\
 \% replaced    & 40.2\%            &39.7\%             \\
		\bottomrule
	\end{tabular}
\end{table}

\subsection{Explicit Features}

% - What we add and how we collected these data.
%   - Give an example of an amendment
%   - Explain what are the explicit (meta) features and the text features.
We extract explicit (meta) features of the MEPs, the edits, and the dossiers, as well as text features.
For each MEP, we collect their nationality (one of 28), their EU political group (one of 8 or 9), and their gender.
A political group clusters national parties that share similar political ideologies.
For each edit, we identify whether it is an insertion, a deletion, or a replacement of some words in the proposal, and we compute its length.
We also collect information about where in the law the edit was proposed: in an article (in the body of the proposal), in a recital (in the preamble of the proposal), in an annex, or in other more specific but less frequent parts of a law.
We determine whether an edit in a reporting committee comes from an opinion committee (in which case it is an ``outsider'').
Finally, we note whether an edit comes with an optional justification.
For each dossier, we identify its type (report or opinion) and the committee that is in charge.
We also note if the proposal is a regulation (legally binding for all member states of the EU), a directive (sets general goals that member states can implement however they want), or a decision (binding to one member state or company only).
We describe these explicit features in Table~\ref{lmp:tab:features}.

\begin{table}
  \centering
	\caption{List of features for MEPs and edits.}
	\label{lmp:tab:features}
	\begin{tabular}{lll}
		\toprule
		Category & Feature            & Type [Values]                     \\
		\midrule
		MEP      & Nationality        & Categorical [28]                  \\
		         & Political group    & Categorical [8 or 9]              \\
		         & Gender             & Categorical [2]                   \\
		% & Age                 & Categorical [8 or 9] \\
		% & Experience          & Categorical [8 or 9] \\
		         & Rapporteur         & Binary                            \\
		Edit     & Edit type          & Categorical [3]                   \\
		         & Log-length (+)     & Numerical [$\mathbf{R}_{\geq 0}$] \\
		         & Log-length (-)     & Numerical [$\mathbf{R}_{\geq 0}$] \\
		         & Article type       & Categorical [7]                   \\
		         & Outsider committee & Binary                            \\
		         & Justification      & Binary                            \\
		Dossier  & Type               & Categorical [2]                   \\
		         & Committee          & Categorical [35]                  \\
		         & Legal act          & Categorical [3]                   \\
		%         & Directorate-General & Binary \\
		%         & Subject matter      & Binary \\
		%         & Length of proposal  & Binary \\
		\bottomrule
	\end{tabular}
\end{table}

\subsection{Text Features}

We further augment the dataset by collecting text features of the edit itself.
It is reasonable to expect that certain words and phrases are predictive of the success of an edit.
We extract the deleted words~$w_-$ from the proposal and the inserted words~$w_+$ from  the amendment.
In Figure~\ref{lmp:fig:amendment}, for example, Edit (b) of Amendment 802 has~$w_- = \textit{``by''}$ and~$w_+ = \textit{``uploaded by users of''}$.
We also consider the context of an edit by extracting the original text of the whole amended article.
For Amendment 802, the context is the portion of text labelled as \textit{``Text proposed by the Commission''}.
Finally, we also extract the title of the law proposal; we will use it as a text feature of the dossier.
For Amendments 802 and 803, the title is \textit{``Copyright in the Digital Single Market''}.
We map all words to lower case, and we replace digits in the title by the letter ``D'', as there are many reference numbers that are unlikely to be useful for our task.

We give some statistics of the distribution of the length of the deleted text~$w_-$, the inserted text~$w_+$, the context, and the title in Table~\ref{lmp:tab:text}.
We report the lower quartile~$Q_1$ and the upper quartile~$Q_3$, as well as the median.
About half of the inserted and deleted texts are short (7 words or less), but the distribution of lengths has a long tail, as shown by the larger values of the upper quartile~$Q_3$.
The context provides large portions of text (the median is at 42 for EP7 and 49 for EP8), which will be useful for making predictions.
In Section~\ref{lmp:sec:models}, we describe how we incorporate the explicit features and the text features into our models.

\begin{table}
  \centering
	\caption{Distribution of text lengths in number of words.}
	\label{lmp:tab:text}
	\begin{tabular}{llrrr}
		\toprule
		Legislature & Type            & $Q_1$ & Median & $Q_3$ \\
		\midrule
		EP7         & Insertion $w_+$ & 2     & 7      & 20    \\
		            & Deletion $w_-$  & 2     & 6      & 26    \\
		            & Context         & 15    & 42     & 79    \\
		            & Title           & 6     & 12     & 19    \\
		EP8         & Insertion $w_+$ & 2     & 6      & 17    \\
		            & Deletion $w_-$  & 2     & 6      & 28    \\
		            & Context         & 20    & 49     & 93    \\
		            & Title           & 6     & 10     & 22    \\
		\bottomrule
	\end{tabular}
\end{table}

%- Describe the text (give some descriptive statistics, some plots, ...).
% - Distribution of edit length
% - How many amendments have justification
% - Most frequent words and bigrams
% - Count identical edits/amendments

%! TEX root = ../thesis.tex
\section{Edit Graph}
\label{lmp:sec:collconf}

We describe the dynamics of the legislative process in terms of the conflicts between edits.
%We take a graph theoretical approach to explain our model in Section~\ref{lmp:sec:models}.
For each dossier, we construct the edit graph $ G = (V_G, E_G) $, such that each node $ v \in V_G $ is an edit and such that there is an undirected edge $ (u, v) \in E_G $ if edits $u$ and $v$ overlap.
A component of size at least~2 in $G$ is therefore a group of overlapping edits. % removed ``connected'' -> it's part of the definition of ``component'', so unnecessary
An isolated node corresponds to an edit that does not overlap with any other edit.

In Figure~\ref{lmp:fig:edit_graph}, we show the edit graphs of three regulations of EP7.
We depict each node with a green dot if the edit is accepted, and with a red cross if the edit is rejected.
The ``transportable pressure equipment'' (left), a very specific legislation, exhibits a graph with 96 nodes, among which 97\% are accepted.
The graph contains only isolated nodes, meaning that no edits overlap: all its components are size 1.
The ``European capitals of culture'' (center), which can affect some cities of member states, exhibits a graph with 58 nodes, among which 48\% are accepted.
The graph contains 16 cliques and the average component size is 1.49.
The GDPR (right), with high stakes for both businesses and consumers, exhibits a graph with 3154 nodes, among which only 9\% are accepted.
The graph contains 1298 cliques, meaning that many edits are conflicting, and has an average component size of 3.44.

\begin{figure}
  \centering
	\newcommand{\imgscale}{0.88}
	\includegraphics[scale=\imgscale]{lmp-edit-graph-tpe}\hfill\includegraphics[scale=\imgscale]{lmp-edit-graph-ecc}\hfill\includegraphics[scale=\imgscale]{lmp-edit-graph-gdpr}
	\caption{
		(Left) The ``transportable pressure equipment'' edit graph contains 96 edits (97\% accepted) and no conflicts.
		(Center) The ``European capitals of culture'' edit graph contains 58 edits (48\% accepted) and 16 conflicts.
		(Right) The GDPR edit graph contains 3154 edits (9\% accepted) and 1298 conflicts.
	}
	\label{lmp:fig:edit_graph}
\end{figure}

\subsection{Conflicts}

Conflicts are inherent in the ordinary legislative procedure defined in Section~\ref{lmp:sec:background}, as every proposed edit reflects a disagreement with the initial law proposal.
A first class of conflicts occur between the proposal and each edit proposed by MEPs.
These conflicts appear as components of any size in $G$.
Hence, every isolated node and every clique in $G$ are such conflicts.
We call them ``conflicts with the status quo'', as they are in disagreement with the proposal.
For example, each edit of Amendments 108 and 5 in Figure~\ref{lmp:fig:amendment} is such a conflict.
In Figure~\ref{lmp:fig:edit_graph} (left), each green node is an edit accepted over the status quo, and each red node is an edit rejected over the status quo.
Similarly, in Figure~\ref{lmp:fig:edit_graph} (center), the cliques with all red nodes are rejected over the status quo.

Another class of conflicts occur between two or more edits proposed by MEPs.
If several MEPs propose different edits on the same part of a text, they compete with each other for the acceptance of their suggestions.
In this case, the edits conflict with the status quo \textit{and} with edits proposed by other MEPs.
These conflicts appear as a clique of size at least~2 in $G$, as there is an edge between overlapping edits.
For example, in Figure~\ref{lmp:fig:amendment}, the first edit in Amendment 108 and the first edit in Amendment 5 form such a conflict.
It corresponds to a clique of size~2.
In Figure~\ref{lmp:fig:edit_graph} (left), there are no such conflicts.
As no edge links any two nodes, all conflicts are only with the status quo.
In Figure~\ref{lmp:fig:edit_graph} (center), however, the cliques with one green node and one or more red nodes are conflicts between several edits, where one edit is accepted over the others and over the status quo.

In $G$, two green nodes cannot appear at both ends of the same edge, as only one edit can be accepted among those that are conflicting.
% Hence, any clique can have at most one green node, and green nodes can only appear as an independent set on the components.
Hence, green nodes can only appear as an independent set on the components.
Two red nodes, however, can appear at both ends of the same edge, as they can both be rejected: this is the case with the first edit in Amendments 108 and 5.

Conflicts between edits can be easily projected to conflicts between MEPs, as we know the authors of each edit.
We compare the conflictive dynamics between MEPs by comparing the distribution of (i) the number of cliques and (ii) the size of cliques in the edit graph $G$ of each dossier.
The median number of cliques in EP7 is 14, which is smaller than 32 in EP8.
The median size of cliques in EP7 is 2.23, which is smaller than 2.38 in EP8.
There are therefore (i) more conflicts and (ii) conflicts of larger size in EP8, compared to EP7.
This increased heterogeneity in the clique structures of edit graph $G$ suggests that predicting the outcome of edits is more difficult for EP8.

\subsection{Collaboration}

Here, we construct the collaboration graph $ H = (V_H, E_H) $ by projecting edits onto the space of MEPs.
Each node $ v \in V_H $ is a MEP and there is a weighted edge $ (u, v) \in E_H $ if MEPs $u$ and $v$ co-sign an edit, where the weights count the collaborations.
The node-degree distribution, i.e., the distribution of number of collaborators, is well fitted by a power-law distribution whose median is 61 for EP7 and 136 for EP8.
Hence, MEPs tend to collaborate with many colleagues in general, and more so in EP8.

We quantify (i) national and (ii) political collaborations by computing the modularity \cite{newman2006modularity} in graph $H$ when defining communities by nationality or by political group.
Modularity is a measure of the strength of the community structure in a graph.
It takes values between $-1$ and $1$, with a higher positive value indicating stronger community structure.
In order to obtain comparable measurements, we merge the two right-wing populist, euroskeptic groups of EP8 to obtain 8 political groups, as in EP7\footnote{Communities of equal size are required to enable fair comparison of modularities. One right-wing populist group in EP7 split into two at the beginning of EP8.}.
We compute the modularity $ Q_n^{(l)} $ when clustering MEPs by nationality in the $l$-th legislature and $ Q_p^{(l)} $ when clustering MEPs by political group.
Computing the modularities in both legislatures, we obtain
\begin{align*}
  Q_n^{(7)} = 0.17  &>  0.05 = Q_n^{(8)}, \\
  Q_p^{(7)} = 0.22  &>  0.18 = Q_p^{(8)}.
\end{align*}
This suggests that political affinity is more important than national affinity to drive collaboration in EP8 compared to EP7.
The political science has not settled on this point: political cohesion is stronger than national cohesion in the EU Parliament in some works \cite{hix2002parliamentary,hix2008voting,mcelroy2010party}, and national cohesion is stronger than political cohesion in other works \cite{cicchi2013logic,hix2013empowerment,cencig2017voting}.
To the best of our knowledge, however, all previous work about political and national cohesion is performed using vote outcome data rather than amendment outcome, an inherently different setting.

% The voting process is simpler than the amending process explained in Section~\ref{lmp:sec:background}.
% MEPs in plenary vote on reported amendments only, and not on the myriad of proposed amendments in the committees.
% Moreover, the outcome of committee votes are digitally registered only if requested by a political group or by at least 40 MEPs (they otherwise vote by show of hands).
% Vote datasets are therefore of smaller size and less rich.

%! TEX root = ../thesis.tex
\section{Statistical models}
\label{sec:models}

We propose a statistical model of edit outcomes from conflicts.
We incorporate assumptions reminiscent of the Bradley-Terry model \cite{bradley1952rank} and of the Rasch model \cite{rasch1960probabilistic}, as follows.
We model the amending process as a "game" between (a) the MEPs themselves (similar to the Bradley-Terry model) and (b) the MEPs and the status quo (similar to the Rasch model).
For simplicity, let us suppose that an edit proposed by MEP $u$ is accepted on dossier $i$ over a conflicting edit proposed by MEP $v$.
As an example, a MEP from one party might propose a modification favoring economic interests, whereas another MEP from another party proposes a modification at the same position in the proposal favoring social interests.
We model the probability of the edit proposed by MEP $u$ to be accepted over the edit proposed by MEP $v$ on dossier $i$, i.e., the  probability of MEP $u$ "winning" over MEP $v$ on dossier $i$ as
\begin{align}
	\label{eq:basemodel}
	p( u \succ_i v )
	 & = \frac{\exp(s_u)}{\exp(s_u) + \exp(s_v) + \exp(d_i + b)} \nonumber  \\
	 & = \frac{1}{1 + \exp[ -( s_u - s_v ) ] + \exp[ -( s_u - d_i ) + b ]},
\end{align}
where $ s_u, s_v \in \mathbf{R} $ are the \textit{skills} of MEPs $u$ and $v$, $ d_i \in \mathbf{R} $ is the \textit{inertia} of dossier $i$, and $ b \in \mathbf{R} $ is a global bias parameter.
The first exponential in the denominator of \eqref{eq:basemodel} encodes the MEP-MEP interaction.
The second exponential encodes the MEP-dossier interaction.
If an edit proposed by MEP $u$ does not conflict with any other edits, the MEP-MEP term vanishes, leaving only the MEP-dossier term.

The parameters in this model enable interpretation.
The skill $s_u$ quantifies the ability of MEP $u$ to pass an edit representing their views.
We interpret a high skill as a high \textit{influence}.
The inertia $d_i$ quantifies the resistance to change of dossier~$i$.
This resistance is not due to the dossier resisting \textit{per se} but rather to the effect of other MEPs voting the edits or proposing conflicting edits.
In this sense, we interpret a high inertia as a sign of possible high \textit{controversy}.
The general bias term $b$ tunes the importance that the model gives to the MEP-MEP term relative to the MEP-dossier term.
We conduct an in-depth analysis of the parameters in Section~\ref{sec:results}.

\paragraph{Multiple Authors and Multiple Conflicts}
As explained in Section~\ref{sec:data} and Section~\ref{sec:collconf}, one or more MEPs can propose an edit, and an edit can be in conflict with one or more other edits.
It is easy to generalize~\eqref{eq:basemodel} to multiple authors and multiple conflicts.
To model multiple authors, we simply sum the skills of each author of an edit.
To model multiple conflicts, we observe that each conflict generates a new MEP-MEP interaction term.
Call \mbox{$\mathcal{C} = \{ a, b, \dots \}$} the set of conflicting edits proposed by authors $ \mathcal{A}_a, \mathcal{A}_b, \dots $.
Note that $ \mathcal{C} $ forms a clique in the edit graph $ G $ of Section~\ref{sec:collconf}.
The probability of edit $a$ being accepted over edits $b, \dots$ on dossier $i$ is given by
\begin{equation}
	\label{eq:multiple}
	p\left( a \succ_i \mathcal{C} - \{ a\} \right) =
	\frac{\exp(s_a) }{ \sum\limits_{c \in \mathcal{C} } \exp(s_c) + \exp(d_i + b) },
\end{equation}
where $s_a = \sum_{u \in \mathcal{A}_a} s_u$ is the cumulated skill of all authors of edit~$a$.
We refer to this model as the \wow\ model.
The probability that all edits are rejected, i.e., the status quo of dossier~$i$ wins, is given by
\begin{equation*}
	p\left(i \succ \mathcal{C} \right)
	= 1 - \sum_{a \in \mathcal{C} } p( a \succ_i \mathcal{C} - \{ a \})
	= \frac{\exp(d_i + b) }{ \sum\limits_{a \in \mathcal{C} } \exp(s_a) + \exp(d_i + b) }.
\end{equation*}


\paragraph{Rapporteur Feature}
We focus on the role of \textit{rapporteur}, explained in Section~\ref{sec:background}.
A rapporteur is a MEP with a special role in shaping a dossier, which plausibly confers additional influence compared to other MEPs.
In order to validate this hypothesis, we add a parameter~$r \in \mathbf{R}$ to the skill~$s_u$ of a MEP~$u$ if they are the rapporteur for the dossier~$i$, i.e., we replace $s_a$ in~\eqref{eq:multiple} by
\begin{equation*}
	s_a = \sum_{u \in \mathcal{A}_a} s_u + r\vec{1}_{\{u \text{ is rapporteur for } i\}}.
\end{equation*}
We refer to this model as the \wowr\ model.

\paragraph{Learning the Model}
Each observation $k$ is a triplet $ ( \mathcal{C}_k, i_k, \ell_k )$ of (a) a set of conflicting edits $ \mathcal{C}_k $ with $ | \mathcal{C}_k | = c_k > 0 $ , (b) a dossier $ i_k $ on which the edits are proposed, and (c) a label $ \ell_k \in \mathcal{C}_k \cup \{ i_k \} $ indicating which of the $c_k$ edits or the status quo is accepted.
Given a dataset of $K$ independent triplets \mbox{$\mathcal{D} = \{ ( \mathcal{C}_k, i_k, \ell_k )~\vert~k = 1, ..., K \}$}, we learn the parameters by maximizing their log-likelihood under $ \mathcal{D} $.
That is, by collecting all the parameters into a single vector $ \vec{\theta} $, we seek to minimize the negative log-likelihood
\begin{align}
	\label{eq:log_likelihood}
	- \ell(\vec{\theta} ; \mathcal{D})
	= \sum_{k = 1}^K  \sum_{a \in \mathcal{C}_k} \Biggl[ & \vec{1}_{\{\ell_k = a\}} \log p\left(a \succ_{i_k} \mathcal{C}_k - \{ a \} \right)  \nonumber \\
	                                                     & + \vec{1}_{\{\ell_k = i_k\}} \log p\left(i_k \succ \mathcal{C}_k \right) \Biggr],
\end{align}
where $ p\left(a \succ_{i_k} \mathcal{C}_k - \{ a \} \right) $ and $ p\left(i_k \succ \mathcal{C}_k \right) $ depend on~$\vec{\theta}$.
In order to avoid overfitting, we add $L_2$-regularization to the negative log-likelihood.
We pre-process our dataset by keeping only the dossiers for which more than 10 edits were proposed and by keeping only the MEPs who proposed more than 10 edits.
Hence, we obtain a dataset of $K=125733$ data points for EP7 and $K=140763$ data points for EP8.
We split them into 70\% for training and validation, and we keep 30\% as a test set.
The log-likelihood~\eqref{eq:log_likelihood} is convex, and we find optimal parameters by using a convex optimizer, such as L-BFGS-B \cite{byrd1995limited}.

%! TEX root = ../thesis.tex
\section{Experimental Results}
\label{lmp:sec:results}

\subsection{Baselines}
% - Baseline:
%   - Random
%   - Advanced random -> adapts to the size of conflict
%   - WoW(.) and WoW(R)
% - Introduce the WoW model (starting from the previous paper); this model will serve as a baseline.
%   - We adopt the terminology of us et al.

We start by introducing the baselines against which we compare our models.
For each baseline and for our models, we assume a set of~$K$ conflicting edits \mbox{$\mathcal{C} = \{ a, b, \ldots \}$} proposed on dossier~$i$, for which we want to model the probability that an edit~$a \in \mathcal{C}$ is accepted over edits~$b, \ldots$ on this dossier.
We denote this probability by~$\Prob{ a \succ_i \mathcal{C} - \{ a\} }$, and we denote the probability that the status quo wins, \textit{i.e.}, that the original text proposed by the Commission is kept, by~$\Prob{ i \succ \mathcal{C} } = 1 - \sum_{a\in \mathcal{C}} \Prob{ a \succ_i \mathcal{C} - \{a\} }$.

\paragraph{Naive Classifier}

The \textit{naive classifier} predicts a uniform probability for each outcome, \textit{i.e.}, for each of the conflicting edits or the status quo to win, as
\begin{equation*}
	\Prob{ a \succ_i \mathcal{C} - \{ a\} } = \Prob{ i \succ \mathcal{C} } = \frac{1}{K + 1}.
\end{equation*}

\paragraph{Random Classifier}

The \textit{random classifier} learns the prior probability~$p^{(K)}$ that the status quo wins for each conflict size~$\vert \mathcal{C} \vert = K$, and it predicts
\begin{equation*}
	\Prob{ i \succ \mathcal{C} } = p^{(K)}.
\end{equation*}
It predicts uniformly each of the edits to win as
\begin{equation*}
	\Prob{ a \succ_i \mathcal{C} - \{ a\} } = \frac{1-p^{(K)}}{K}.
\end{equation*}

\subsection{Experimental Setting}

% Explain the metrics we use for evaluation.
We report the cross-entropy loss to evaluate the baselines and our models.
Let~$( \mathcal{C}_n, i_n, l_n )$ be an observation.
We compute
\begin{equation}
	\ell_n = \begin{cases}
		-\log p(l_n \succ_{i_n} \mathcal{C}_n - \{l_n\} ) & \text{if $l_n \in \mathcal{C}_n$}, \\
		-\log p(i_n \succ \mathcal{C}_n)                  & \text{if $l_n = i_n$}.
	\end{cases}
\end{equation}
We report the average value for all~$N$ points in our test set as~$\ell = \frac{1}{N} \sum_n \ell_n$.
%, and we compute the 99\% confidence intervals around this average.
% - Explain the training and testing procedure
% - Explain how we chose the best hyperparameters
We randomize our dataset and we split it into 80\% for training, 10\% for validation, and 10\% for the final evaluation.
Note that an edit can be involved in several conflicts.
For example, in Figure~\ref{lmp:fig:amendment}, edit~$c$ is in involved in two conflicts: $\mathcal{C}_1 = \{ c, d \}$ and $\mathcal{C}_2 = \{ c, e \}$.
Hence, we assign conflicts to each set so that an edit is present in exactly one set.
We combine both the training and the validation sets to fit our model before evaluating it on the test set.
We set the number of latent dimensions~$L$ and the regularizers, and we choose the best word embeddings, by held-out validation.
% Please refer to Appendix~\ref{app:hyperparameters} for a complete list of the best hyperparameters that we used.
This results in fastText of dimension~$D = D' = 10$, with bigrams.

\subsection{Predictive Performance}
% - Give and discuss the predictive performance of all models
%   - Predicting a new edit
% - Plot the results in terms of
%   - Log-loss
%   - Accuracy?
%   - F1-Score/AUC?

We show in Figure~\ref{lmp:fig:results} the overall performance of all variations of our model (with and without explicit, latent, and text features) over EP7 and EP8, and we compare them against the naive and the random predictors, as well as against the \wow{}\ model.
All our models outperform the baselines, and \wow{XLT} outperforms all other models.
Including explicit features improves the performance of the predictions in terms of the cross entropy by 7\% for EP7 and 6\% for EP8 over the simpler \wow{}\ model.
On EP7, \wow{L}\ improves the performance by 12\% and \wow{T} by 7\%, whereas for EP8 the difference between the two models is smaller (10\% increase for \wow{L}\ and 8\% for \wow{T}).
Hence, the text features provide a greater improvement for EP8 than for EP7, while the latent features provide a greater improvement for EP7 than for EP8.
The difference between \wow{XL}\ and \wow{L}\ (0.010 for EP7 and 0.013 for EP8) is less than the difference between \wow{XT}\ and \wow{T}\ (0.034 for EP7 and 0.035 for EP8), as the latent features absorb the effects of the explicit features more than the text features do.
Finally, combining the text and latent features provides high performance, but further combining them with explicit features leads to the best performance.

\begin{figure}
	\centering
	\includegraphics{lmp-results}
	\caption{%
		Average cross-entropy loss of the baselines and our models.
		Combining the explicit, latent, and text features help obtain the best performance.
	}
	\label{lmp:fig:results}
\end{figure}

\subsection{Error Analysis by Conflict Size}
% - Compute the performance of high- and low-controversy dossiers
%   - Dynamics of law-making is different depending of the level of controversy
%   - We can not use the same predictor, but the hybrid models bridges the gap
% - Compute performance per committees
% - Compute performance per topic

We explore how the \wow{XLT}\ model performs on conflict of different sizes in the test set for EP8 (we observe a similar behaviour on EP7).
We bin the conflict size so that there are at least 100 data points in each bin.
The distribution of conflict size is exponentially decreasing: There are 8462 conflicts of size 1 (i.e., an edit is in conflict with the status quo only), 3063 conflicts of size 2 (i.e., two edits are in conflict, as well as with the status quo), and 140 conflicts of size 7 and more.
We compare the average cross entropy of the \wow{XLT}\ model with that of the random predictor and that of the \wow{}\ model.
In Figure~\ref{lmp:fig:error-analysis}, we see that while the loss generally increases with conflict size for all three models, it increases less rapidly for the \wow{XLT}\ model than for the \wow{}\ model.
This suggests that the explicit, latent, and text features enable the model to exploit the increasing complexity of data points to make more accurate predictions.
We also see that for conflicts of size 4 and higher, the \wow{}\ model performs worse than the random predictor, but the \wow{XLT}\ model is able to outperform it.

\begin{figure}
	\centering
	\includegraphics{lmp-error-analysis}
	\caption{%
		Average cross-entropy loss per conflict size~$\vert \mathcal{C} \vert = K$.
		The loss of the \wow{XLT} model increases less rapidly than the loss of the baselines.
	}
	\label{lmp:fig:error-analysis}
\end{figure}

\subsection{Contribution of Explicit Features}

% - Explore the contribution of different explicit features to the performance of the model
To understand the contribution of the explicit features to the predictive performance, we show in Figure~\ref{lmp:fig:improvement} the decrease in cross-entropy loss of \wow{MEP}\ (all MEP features but the rapporteur feature), \wow{Rapporteur}\ (rapporteur feature only), \wow{Edit}, and \wow{Dossier} over \wow{}.
The dossier features contribute virtually nothing to the predictive performance (the difference is at the fourth decimal point).
Similarly, for EP7, the nationality, political group, and gender features of \wow{MEP} contribute very little.
For EP8, these features improve the performance, but not as much as the edit features.
This suggests that these features have limited influence on the predictions.
Nationalities and political groups have been qualitatively analyzed in the literature in the context of their influence on MEPs' voting behaviour~\cite{hix2002parliamentary,coman2009reassessing,muhlbock2012national,lefkofridi2014multilevel}.
To the best of our knowledge, there is no analysis of their effect on the amending process.
Interestingly, for EP7, combining all features into the \wow{X}\ model leads to a performance boost that is greater than the sum of each individual feature groups.

\begin{figure}
	\centering
	\includegraphics{lmp-improvement}
	\caption{%
		Difference in cross-entropy loss over \wow{} of different models.
		The rapporteur feature and the edit features contribute more to the predictive performance than the MEP and dossier features.
	}
	\label{lmp:fig:improvement}
\end{figure}

\subsection{Interpretation of Explicit Features}

To get insights into the dynamics of the legislative process, we interpret the values of the parameters of \wow{XLT} trained on the full dataset for EP8 (combining training, validation, and test data).
Let~$w_f \in \mathbf{R}$ be the value of the parameter associated with feature~$f$.
The rapporteur feature~$r$ of \wow{Rapp.}\ provides a greater decrease in loss.
This \textit{rapporteur advantage} complements the findings of~\citet{costello2010policy}, conducted by interviewing key informants over EP5 (1999-2004) and EP6 (2004-2009).
They show that the rapporteur, with their particular role, has some influence on the legislative process, albeit constrained.
We note that, according to our model, the rapporteur advantage has slightly increased in EP8 ($w_r=1.19$) compared to EP7 ($w_r=1.12$).

% - Interpret the values that different parameter of the explicit features take
These explicit features enable us to explain what contributes to the success of an edit.
We report here (and in subsequent sections) the results for EP8 only.
All other things being equal, a female ($w_{\text{fem}}=-0.02 > -0.04 = w_{\text{mal}}$) MEP from Latvia and whose party belongs to the group of the European People's Party (center-right) has the highest chance to see her edit accepted.
This edit has even higher chances if it inserts ($w_{\text{ins}}=-0.03 > w_{\text{del}}=-0.13 > w_{\text{rep}}=-0.22$) a short portion of text (the feature associated with both insertion and deletion length is negative) in a part of the law that is not its body or its preamble ($w_{\text{art}}$, $w_{\text{rec}}$  and~$w_{\text{para}}$ have the lowest value among the seven article types).
Adding a justification also increases the probability of an edit being accepted ($w_{\text{jus}}=0.08$), as well as edits from the opinion committee (referred to as the ``outsider committee'' feature in Table~\ref{lmp:tab:features},~$w_{\text{out}} =  0.16$).

For the dossier features, our model learns that it is harder to make edits on reports, as compared to opinions ($w_{\text{rep}}=0.33 > -0.26 = w_{\text{opi}}$).
As explained in Section~\ref{lmp:sec:dataset}, reports are voted by the whole Parliament.
Therefore, they have a greater influence on the final law, and we expect that MEPs make it more difficult for competing edits to be accepted in reports.
Finally, our model also learns that it is harder to make edits for decisions and directives, as compared to regulations ($w_{\text{dec}}=0.25 > w_{\text{dir}} = 0.12 > w_{\text{reg}} =0.10$).

\paragraph{Controversy of Dossiers}

Table~\ref{lmp:tab:inertia_params} provides a list of the ten dossiers in EP8 with the highest inertia parameter~$d_i$ and the ten dossiers with the lowest~$ d_i $.
Overall, the values of $d_i$ correlate well with the number of nodes, the number of cliques, the average size of cliques, and the edit acceptance rate.
These four metrics are a good proxy to the level of activity by MEPs in the amending process of a given dossier.
Higher activity, possibly due to higher controversy, leads to higher value of~$d_i$.
We note, however, that some of the top-10 dossiers have a small number of edits.
This shows that the inertia parameters capture more information than simply some of these descriptive statistics.

The top-ten dossiers include laws with high stakes about financial markets, the environment, vast investment programmes, and assistance to member states:
The ``Screening of foreign direct investments'' sets a framework to better equip the EU for investments from non-EU countries.
It has crucial implications for companies, workers, governments, and citizens.
The ``European Supervisory Authorities on financial markets'' sets strict regulations for the financial markets.
``InvestEU'' and the ``Horizon Programme'' are vast investment programmes for innovation and research.
The ``Cost-effective emission reductions and low-carbon investments'' is one of the implementations of the Paris Climate Agreement.
Finally, The infamous ``Copyright in the Digital Single Market'', considered to be a threat to freedom of expression on the Web by its opponents, sparked public protests in several cities.
The reporting committee publicized that ``MEPs have rarely or never been subject to a similar degree of lobbying before''~\citep{europarl2019questions}.

\begin{sidewaystable}
	\centering
	\caption{Top-10 and bottom-10 inertia parameters~$d_i$ for dossiers in EP8.}
	\label{lmp:tab:inertia_params}
	\begin{tabular}{rlllrrrr}
		\toprule
		$d_i$  & Type & Comm. & Title                                                         & \# edits & \# conf. & avg.\ cf.\ sz. & \% acc. \\
		\midrule

		2.018  & Rep. & INTA  & Screening of foreign direct investments                       & 1040     & 272      & 3.1            & 2.6     \\
		1.958  & Opi. & ITRE  & Cost-effective emission reductions and low-carbon investments & 1756     & 385      & 4.2            & 5.1     \\
		1.879  & Opi. & PETI  & Discontinuing seasonal changes of time                        & 81       & 25       & 2.9            & 6.2     \\
		1.619  & Rep. & ENVI  & Health technology assessment and amending                     & 133      & 14       & 2.0            & 4.5     \\
		1.512  & Rep. & ECON  & European Supervisory Authorities on financial markets         & 48       & 12       & 2.2            & 10.4    \\
		1.447  & Rep. & ECON  & InvestEU Programme                                            & 1194     & 297      & 2.9            & 27.0    \\
		1.393  & Rep. & ITRE  & Horizon Europe                                                & 2013     & 467      & 3.0            & 9.8     \\
		1.386  & Rep. & INTA  & Macro-financial assistance to the Republic of Moldova         & 36       & 8        & 2.4            & 13.9    \\
		1.286  & Rep. & AFET  & Instrument for Pre-Accession Assistance                       & 732      & 239      & 2.5            & 20.6    \\
		1.282  & Rep. & JURI  & Copyright in the Digital Single Market                        & 2657     & 577      & 4.3            & 2.6     \\

		\midrule

		-1.651 & Opi. & REGI  & Common agricultural policy                                    & 105      & 4        & 2.0            & 82.9    \\
		-1.655 & Opi. & DEVE  & Promotion of the use of energy from renewable sources         & 62       & 3        & 2.0            & 90.3    \\
		-1.681 & Opi. & AGRI  & Establishing Horizon Europe                                   & 43       & 8        & 2.0            & 65.1    \\
		-1.686 & Opi. & AGRI  & Governance of the Energy Union                                & 150      & 30       & 2.3            & 56.7    \\
		-1.754 & Opi. & JURI  & Insurance against civil liability with motor vehicles         & 29       & 2        & 2.0            & 89.7    \\
		-1.779 & Opi. & BUDG  & Common agricultural policy                                    & 15       & 0        & 0.0            & 100.0   \\
		-1.780 & Opi. & AGRI  & European Regional Development and Cohesion Fund               & 129      & 13       & 2.2            & 58.1    \\
		-1.812 & Opi. & ECON  & Prevention and prosecution of criminal offences               & 81       & 2        & 2.0            & 86.4    \\
		-2.065 & Opi. & DEVE  & Unfair trading practices in in the food industry              & 63       & 6        & 2.0            & 84.1    \\
		-2.284 & Opi. & TRAN  & Protection of the collective interests of consumers           & 121      & 26       & 2.1            & 66.9    \\

		\bottomrule
	\end{tabular}
\end{sidewaystable}

\subsection{Interpretation of Text Features}
\label{lmp:sec:intertext}

In Figure~\ref{lmp:fig:results}, we observe that the text features contribute significantly to improving the performance.
We use the learned parameter vectors~$\boldsymbol{w}_T$ and~$\boldsymbol{w}_{T'}$ of \wow{XLT}\ to identify words and bigrams that have the most predictive power.
First, we rank the words and bigrams of the edit text, according to the dot product of their embeddings with~$\boldsymbol{w}_T$.
The top-$k$ terms (having a positive dot product) contribute the most towards acceptance of the edit, whereas the bottom-$k$ terms (having a negative dot product) contribute most towards rejection of the edit.
The opposite holds for the terms of the title and their dot product with~$\boldsymbol{w}_{T'}$.

We look at the top~$50$ terms for each feature and prediction outcome and find some interesting patterns among these terms, although not all of them are easy to interpret.
Note that we have more than~\numprint{10000} unique terms for the edit text and more than~\numprint{1000} unique terms for the title, hence we consider only the most predictive terms near the ends of the ranking.
% A list of the top-50 terms for each feature and prediction outcome is reported in Appendix~\ref{app:accept}.

%We first examine the words and bigrams in the inserted and deleted text are predictive of acceptance.
%We see the word \textit{consumer} here, which commonly occurs in laws on consumer rights.
%This suggests that deleting provisions of the laws that give rights or benefits to consumers might be getting accepted often (possibly under the influence of lobbies), and indeed we see many such examples in the dataset.
One of the bigrams that, when deleted, is predictive of acceptance is \textit{any other}, which is commonly used to widen the scope of the law (as in ``contractual or any other duty'').
Interestingly, the bigrams \textit{human rights} and \textit{data protection} are also predictive of acceptance when deleted.
The word \textit{should}, which is used to add recommendations, is predictive of acceptance when inserted, while adding \textit{must}, which is used for obligations, is predictive of rejection.
We see that \textit{best} is predictive of acceptance, which is commonly used to make a requirement stronger (as in ``best available scientific evidence'', ``best possible way'').
Adding \textit{positive} and \textit{positive impact} predicts acceptance, whereas adding \textit{negative} predicts rejection.
Adding the word \textit{inserted}, which commonly refers to inserting new articles in existing laws, is predictive of acceptance, whereas \textit{deleted} is predictive of rejection.

Considering the words in the context, we see that \textit{firearms}, \textit{resettlement}, \textit{terrorist} and \textit{fingerprints} are predictive of rejection.
This could be because the laws related to these topics are controversial, hence many edits are rejected due to conflicts.
For the words in the title, we see that \textit{customs}, \textit{community}, \textit{financial}, \textit{fisheries}, and \textit{general budget} are predictive of acceptance, whereas \textit{market}, \textit{framework}, \textit{structural reform}, \textit{emission},  and \textit{greenhouse gas} are predictive of rejection.
This suggests the relative ease or difficulty of editing laws related to these topics, and it correlates well with the values of the difficulty parameters~$d_i$:
The top-50 dossiers with the highest difficulty parameters contain high-controversy dossiers about establishing frameworks for the screening of foreign investments and vast public investment programs (InvestEU and Horizon Europe), as well as regulation of the financial market, copyright in the digital market, and carbon-emission reduction.
The bottom-50 dossiers with the lowest difficulty parameters contain low-controversy dossiers about cohesion within the EU, financial rules, fisheries, and the community code on visas.

\subsection{Interpretation of Latent Features}
% - Interpretation of the latent features
%   - In terms of ideological space for the dossier and the MEPs
%   - Explains why they improve the performance, by capturing different ideologies rather than explicit party assignment

The latent features improve the predictions overall and help capture the complex dynamics of the legislative process.
The best number of latent dimensions is~$L = 20$  for the models including latent features.
In order to interpret the latent features, we gather the latent vectors~$\bm{y}_i$ learned by \wow{XLT}\ into a matrix~$Y = [ \bm{y}_i ]$.
We apply principal component analysis and keep the top-10 and bottom-10 dossiers from each of the first two principal components in EP8.
We use t-SNE~\cite{maaten2008visualizing} to represent these forty dossiers in a two-dimensional space, and we show the projection in Figure~\ref{lmp:fig:tsne}.

We distinguish four clusters.
The cluster at the top-left contains dossiers about fuel quality, renewable energy, trade of animals, and sustainable investments.
It also contains dossiers about electronic communications, the processing of personal data, and sharing public information.
We interpret this cluster as \textit{environment and communications}, and we highlight with green triangles the corresponding dossiers.
The cluster at the top-center contains dossiers about the establishment of defense funds, the prosecution of criminal offenses, and the identification of criminals between member states.
It also contains dossiers about the protection of workers, businesses, refugees, internal markets, and cultural goods.
We interpret this cluster as \textit{defense and protection} (red crosses).
The cluster at the top-right contains dossiers about vast investment and development programmes, finance, and the development of internal markets.
We interpret this cluster as \textit{investment and development} (blue dots).
Finally, the cluster at the bottom-left contains dossiers about economic competitiveness and innovation, as well as frameworks for business development and the funding of start-up companies.
We interpret this cluster as \textit{business and innovation} (orange squares).

\begin{figure}
	\centering
	\includegraphics{lmp-tsne}
	\caption{Visualization with t-SNE of the top-10 and bottom-10 dossiers on the first two principal components in EP8.
		There are four clusters:
		Environment and Communications, Defense and Protection, Investment and Development, and Business and Innovation.
	}
	\label{lmp:fig:tsne}
\end{figure}

\subsection{Solving the Cold-Start Problem}
\label{lmp:sec:cold-start}

% - Predicting an edit for a new dossier
We explore how to solve the cold-start problem by defining a second predictive problem:
Given a dossier~$i$ \textit{for which we have never seen an edit}, and given a conflict~$\mathcal{C} = \{a, b, \ldots \}$, we want to predict which of the edits or the status quo wins.
We order the dossiers by the date a committee received a proposal, and we use the dossiers that contain the first 80\% of the conflicts as a training set.
We use the next 10\% as validation set, and we keep the last 10\% aside as test set.
We ensure that no edits in the training set leak into the validation and test sets.
This scenario is more realistic because we make predictions about new dossiers that the model has never observed before.

We report, in Table~\ref{lmp:tab:newdossier}, the results for \wow{Explicit}, \wow{Text}, and \wow{XT}, together with the baselines.
The latent features cannot be used for this task, as the dossier embeddings~$\vec{y}_i$ are unavailable for new dossiers.
For our models, the difficulty parameter~$d_i$ is set to the average difficulty learned in the training set.
The random predictor, which learns the prior probability of the status quo winning for each conflict size, performs the best out of all the baselines, and it outperforms \wow{Text}.
Our approach outperforms only the random predictor when including explicit features.
This suggests that the dossier features help us make more accurate predictions by learning parameter values for the type of dossier, its legal act, and its committee in charge.
In this case, adding text features further boosts the performance.

The overall performance, however, is mixed:
The improvement of \wow{XT} over the random predictor is rather small.
One possible explanation is that the legislative process might be non-stationary.
Hence, our model overfits on the training set, which is very different from the test set.
The task is also unfair to our model, as in a real setting, predictions would be made for the next dossier only.
In the current setting, we make predictions for all future dossiers.
We keep further investigations of this aspect for future work.

\begin{table}
	\centering
	\caption{Average cross entropy of the baselines and our model on predicting new, unseen dossiers.}
	\label{lmp:tab:newdossier}
	\begin{tabular}{llr}
		\toprule
		Type     & Model          & Avg.\ cross entropy \\
		\midrule
		Baseline & Naive          & 0.947               \\
		         & Random         & \textbf{0.800}      \\
		         & \wow{}         & 0.873               \\
		\midrule
		Ours     & \wow{Explicit} & 0.784               \\
		         & \wow{Text}     & 0.839               \\
		         & \wow{XT}       & \textbf{0.759}      \\
		\bottomrule
	\end{tabular}
\end{table}

%! TEX root = ../thesis.tex
\section{Related Work}
\label{lmp:sec:relwork}

%- Cite papers about text of amendments and peer-production systems
Amendment analysis in the European Parliament has been studied by the political science community on datasets of small size~\cite{kreppel1999affects,tsebelis2001legislative,kreppel2002moving,baller2017specialists}.
The effect of the rapporteur on the success of an amendment has been studied in previous legislature periods and in specific committees~\cite{finke2012proposal,hurka2013changing}.
Predicting edits on collaborative corpora of documents has been studied in the context of peer-production systems, such as Wikipedia~\cite{druck2008learning,adler2007content,sarkar2019stre} and the Linux kernel~\cite{jiang2013will,yardim2018can}.
A whole body of literature covers the conflicts between two Wikipedia edits \cite{sumi2011edit,yasseri2012dynamics} and the quantification of controversy of Wikipedia articles \cite{sepehri2012leveraging,rad2012identifying}.
The notion of conflict is, however, different in our setting, where multiple edits can be in conflict at the same time:
The task of predicting which edit will be accepted out of all the conflicting edits is more complex, and classic approaches cannot be used.
In this work, we take a peer-production viewpoint on the law-making process and propose a model of the acceptance of the legislative edits.
Our approach generalizes to any peer-production system in which (meta) features of the users and items can be extracted and in which edits can be in conflict with one another.

We use the text of the edits and dossiers as features for classification.
Text classification is a well-studied problem in natural language processing.
A simple baseline is to apply linear classifiers to term-frequency inverse document-frequency (TF-IDF) vectors~\cite{joachims1998text}.
However, these models do not capture the synonymy relation between words, hence suffer from poor generalization.
Models based on neural networks show better performance on this task~\cite{zhang2015character}.
They tend, however, to require larger datasets, and the features they learn are harder to interpret.
The fastText model~\cite{joulin2017bag} bridges the gap between the two:
It learns embeddings from linear models.
We adapt this approach to our problem of edit classification, as edits are inhomogeneous pieces of text.
Edit modelling has been studied using neural models\cite{yin2018learning,guu2018generating} that suffer from the aforementioned issues of dataset size and interpretability.
In the \warofwords{} models, we combine text features and non-text features to take into account the dynamics of the legislative process.
Legal texts also have features and structures that set them apart from other domains.
For example, the word "should" has a strong legal significance, whereas it is commonly removed as a stop word.

Our model draws inspiration from probabilistic models of choice, described in Section~\ref{in:sec:models}.
First, it borrows from the logit model to model the competitive dynamics between MEPs.
These approaches learn a real-valued score for individuals and model the probability that one individual wins over another as a function of the difference of their scores.
Second, it borrows from the Rasch model to model the competitive dynamics between MEPs and the status quo.
These approaches learn a real-valued strength for each individual and a real-valued difficulty for each item, and they model the probability that an individual wins over the item as a function of the difference of the strength and the difficulty.
Our model unifies both approaches by learning a strength for each MEP and a difficulty for each dossier, considering (i) conflicts between MEPs and (ii) conflicts between MEPs and the status quo.

%! TEX root = ../thesis.tex
\section{Summary}
\label{sec:conclusion}

In this chapter, we have introduced a new dataset of legislative edits and a model of edit outcomes.
Our dataset provides rich information on a long-term, dynamical process of interactions between parliamentarians.
Our proposed model learns a skill parameter for MEPs who propose edits and an inertia parameter for the law proposals that resist to change.
Our model also incorporates (a) explicit features of the edits, of the MEPs, and of the dossiers, (b) latent features of the MEPs and dossiers, and (c) text features of the edits and dossiers.
Each of the three classes of additional features improve the performance significantly, and the best performance is achieved by combining all features.
We interpreted the values of the learned parameters to gain insights into the legislative process.
We provided interpretation of all explicit features to characterize what makes the success of an edit more likely.
We have shown that the latent features capture the representation of MEPs and dossiers in an ideological space.
We have analyzed the words and bigrams in different parts of an edit and a dossier in terms of their influence on the acceptance probability.
We have also analyzed the performance of our model on subsets of the test set based on conflict size, and we have shown that our best model can leverage the features of the data to make more accurate predictions on conflicts of higher size than other baselines.
Finally, we have described how to use our model for predicting edits made on new, unseen dossiers.

\paragraph{Ethical Considerations}
After submitting our paper~\citep{kristof2021war} to the Web Conference 2021, one anonymous reviewer expressed concerns regarding the use of machine learning for making decisions in law making, and whether our findings in Section~\ref{sec:results} could help adversarial attacks.
However, we do not propose to rely on our models for making decisions, such as whether an edit should be accepted or not.
Our goal is to understand the factors correlated with the acceptance of edits, and thereby gain insights into the law-making processes.
These correlations do not imply a causal relationship that would benefit potential adversarial attackers.
\paragraph{Applications and Broader Impact}
We believe that approaches such as ours are helpful to political scientists, journalists and transparency observers, and to the general public:
First, it could be useful in validating theoretical hypotheses using large-scale datasets and advanced computational methods.
Second, it could help uncover lesser-known facts, such as controversial dossiers that slipped under the radar.
Finally, the greater transparency that results from these insights can enhance trust in public institutions and strengthen democratic processes.

% Nevertheless, even if such a relationship were to exist, we prefer that these findings are published in an open research community, where possible countermeasures to such attacks could be thought of for the public good, rather than them being discovered by a company or an influence group that might use them in an opaque manner to push private interests.

\paragraph{Perspective}
First, we currently use pre-trained word embeddings and embeddings trained on an ad-hoc binary classification task.
We plan to explore how to learn text embeddings in an end-to-end manner using the conflictive structure of the \warofwords{} model.
Second, as shown in Section~\ref{sec:cold-start}, our model has only limited predictive power on edits made on future dossiers.
We plan to further explore how to exploit the temporality of the data and how to develop a dynamical model able to take into account the non-stationarity of the law-making process.
Finally, the current setting of the predictive task assumes that conflicts are independent of each other; because an edit can be involved in multiple conflicts, they are not always independent.
We plan to develop more advanced models by leveraging these correlations between conflicts.
For example, we plan to explore how to include latent features and text features to the mixed logit model~\citep{hensher2003mixed}.

\input{9-suppmat}

\clearpage
\bibliographystyle{ACM-Reference-Format}
\bibliography{abbreviations,war-of-words}

\end{document}
