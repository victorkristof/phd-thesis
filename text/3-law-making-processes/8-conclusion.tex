%! TEX root = ../thesis.tex
\section{Summary}
\label{sec:conclusion}

In this chapter, we have introduced a new dataset of legislative edits and a model of edit outcomes.
Our dataset provides rich information on a long-term, dynamical process of interactions between parliamentarians.
Our proposed model learns a skill parameter for MEPs who propose edits and an inertia parameter for the law proposals that resist to change.
Our model also incorporates (a) explicit features of the edits, of the MEPs, and of the dossiers, (b) latent features of the MEPs and dossiers, and (c) text features of the edits and dossiers.
Each of the three classes of additional features improve the performance significantly, and the best performance is achieved by combining all features.
We interpreted the values of the learned parameters to gain insights into the legislative process.
We provided interpretation of all explicit features to characterize what makes the success of an edit more likely.
We have shown that the latent features capture the representation of MEPs and dossiers in an ideological space.
We have analyzed the words and bigrams in different parts of an edit and a dossier in terms of their influence on the acceptance probability.
We have also analyzed the performance of our model on subsets of the test set based on conflict size, and we have shown that our best model can leverage the features of the data to make more accurate predictions on conflicts of higher size than other baselines.
Finally, we have described how to use our model for predicting edits made on new, unseen dossiers.

\paragraph{Ethical Considerations}
After submitting our paper~\citep{kristof2021war} to the Web Conference 2021, one anonymous reviewer expressed concerns regarding the use of machine learning for making decisions in law making, and whether our findings in Section~\ref{sec:results} could help adversarial attacks.
However, we do not propose to rely on our models for making decisions, such as whether an edit should be accepted or not.
Our goal is to understand the factors correlated with the acceptance of edits, and thereby gain insights into the law-making processes.
These correlations do not imply a causal relationship that would benefit potential adversarial attackers.
\paragraph{Applications and Broader Impact}
We believe that approaches such as ours are helpful to political scientists, journalists and transparency observers, and to the general public:
First, it could be useful in validating theoretical hypotheses using large-scale datasets and advanced computational methods.
Second, it could help uncover lesser-known facts, such as controversial dossiers that slipped under the radar.
Finally, the greater transparency that results from these insights can enhance trust in public institutions and strengthen democratic processes.

% Nevertheless, even if such a relationship were to exist, we prefer that these findings are published in an open research community, where possible countermeasures to such attacks could be thought of for the public good, rather than them being discovered by a company or an influence group that might use them in an opaque manner to push private interests.

\paragraph{Perspective}
First, we currently use pre-trained word embeddings and embeddings trained on an ad-hoc binary classification task.
We plan to explore how to learn text embeddings in an end-to-end manner using the conflictive structure of the \warofwords{} model.
Second, as shown in Section~\ref{sec:cold-start}, our model has only limited predictive power on edits made on future dossiers.
We plan to further explore how to exploit the temporality of the data and how to develop a dynamical model able to take into account the non-stationarity of the law-making process.
Finally, the current setting of the predictive task assumes that conflicts are independent of each other; because an edit can be involved in multiple conflicts, they are not always independent.
We plan to develop more advanced models by leveraging these correlations between conflicts.
For example, we plan to explore how to include latent features and text features to the mixed logit model~\citep{hensher2003mixed}.
