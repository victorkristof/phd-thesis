%! TEX root = ../thesis.tex
\section{Conclusion}
\label{sec:conclusion}

In this paper, we have introduced a new dataset of legislative edits and a model of edit outcomes.
Our dataset provides rich information on a long-term, dynamical process of interactions between parliamentarians.
Our proposed model learns a skill parameter for MEPs who propose edits and an inertia parameter for the law proposals that resist to change.
We have provided an interpretation of the parameters, in terms of the influence of MEPs and of the controversy of the laws.
We have also shown that MEPs in the role of rapporteur, hence in charge of a particular dossier, have more influence than other MEPs on the committee.

\paragraph{Future Work}
First, a limitation of our approach is that our model is agnostic to the actual text of the edits.
A cosmetic edit correcting a typo is obviously not equivalent to a more substantial change of the law.
It is however complex to discriminate these two types of edits, as even one word can have critical legal implications (e.g., "shall" versus "should" in the example of Figure~\ref{fig:amendment}).
We plan to investigate this aspect more deeply.
Second, the inclusion of the rapporteur feature, and its subsequent improvement in predictive performance, opens the perspective of including additional features related to the MEPs, the edits, and the dossiers.
This would help improve the performance of our model and better understand what contributes to the success of edits.
Finally, our model assumes that if MEP $u$ is more influential than MEP $v$, then $ p( u \succ_i v ) > p( v \succ_i u ) $ for all dossiers $i$.
This strong assumption is clearly not always realistic: dossiers span a vast amount of different topics, and MEPs have their own specializations and interests.
We plan to improve our model by capturing these dependencies.

% On a more applied side, we believe that it would be valuable to the general public to have easy access to our data.
% The legislative process is a complex process, and its documents are hardly accessible to non-experts.
% In an effort to create transparency and trust, we plan to develop an interactive platform for visualizing our dataset.
