%! TEX root = ../thesis.tex
\section{Introduction}
\label{sec:intro}

The past decade has seen the emergence of several open-government initiatives, such as the one proposed by the former US president, Barack Obama, on the first day of his presidency \cite{open2009barack}.
Similar efforts followed in France, Belgium, Sweden, the United Kingdom, Tunisia, and the European Union.
Open-government data published on the Web are of great interest to citizens, companies, sub- and supra-government entities, and researchers.
These initiatives increase the transparency and trust associated with government, and enable novel analyses to be carried out about their processes.

We are specifically interested in the process by which the laws of a jurisdiction are maintained and in how they evolve over time.
Not surprisingly, the dynamics of the legislative process is complex, given the confluence of many stakeholders, topics, special interests, and lobbying groups.
Until open-government was introduced, the work of parliaments had not been systematically accessible to the general public, and internal documents -- when they existed -- were difficult to find.
We propose (i) a new dataset of legislative edits obtained via rich, openly accessible data from the European Parliament, and (ii) a methodology to better understand the dynamics of legislative processes.
Using edits proposed by parliamentarians on legislative texts, we study the competitive dynamics of collaborations and conflicts between parliamentarians.

We curate our dataset from the European Parliament's online document repository.
It is composed of edits, proposed by parliamentarians, on laws under consideration by the Parliament.
Each data point consists of edit metadata, such as the nationality and the political affiliation of its author(s), the type of edit, its length, and which law it is modifying.
The dataset contains \numprint{449493} edits proposed by \numprint{1214} parliamentarians on \numprint{1889} dossiers over ten years (two legislature periods).
In Section~\ref{sec:background}, we set the framework by giving some background on the European legislative process.
In Section~\ref{sec:data}, we describe our dataset in detail.
In Section~\ref{sec:collconf}, we use our dataset to describe the evolution of a law via a graph-theoretical viewpoint.

Our model focuses on the interplay of collaboration and competition between parliamentarians as they modify laws.
They can collaborate on a proposed modification of a law by jointly submitting an {\em edit} for consideration.
An important feature of our model accounts for the way an edit benefits from the support of multiple parliamentarians.
We posit a measure of {\em strength} for each parliamentarian, and an edit inherits the strengths of its supporters.
There are two sources of competition in the process.
First, a proposed edit competes with the status quo, because the edit can be rejected in favor of not changing the existing state of a law.
Our model incorporates this by endowing each law with a measure of {\em inertia} that represents the level of controversy of a law.
Second, proposed edits of a law are frequently mutually exclusive, because they overlap and are incompatible.
These edits then compete against each other, as well as against the status quo.
This parsimonious set of assumptions underlies our model, formulated in Section~\ref{sec:models}; and we will show, in Section~\ref{sec:results}, that it is sufficient to capture the salient features of the law-making dynamics.
