%! TEX root = ../thesis.tex
\section{Introduction}
\label{sec:intro}

The process of maintaining a body of law in a democratic society shares many features with peer-production systems.
The work of parliaments is governed by complex rules, processes, and conventions, in order to foster compromises among competing viewpoints and priorities.
How well this process works, to what extent it is subject to biases and to benign or undue influences is of obvious concern to citizens and to scientists alike.
An exciting recent development in this regard is the adoption of {\em open government initiatives}, such as in the United States~\citep{open2009barack}, Switzerland~\citep{switzerland2021open}, Brazil~\citep{brazil2021dado}, and the European Union~\citep{european2021data}.
Open-government data published on the Web are of great interest to citizens, companies, sub- and supra-government entities, and researchers.
These initiatives aim to improve the transparency of the law-making process and the accountability of its protagonists.

Not surprisingly, the dynamics of this process is complex, given the confluence of many stakeholders, topics, special interests, and lobbying groups.
Until open-government was introduced, the work of parliaments had not been systematically accessible to the general public, and internal documents -- when they existed -- were difficult to find.
The European Union (EU), however, has been a pioneer in opening the mechanics of its parliament.
It publishes detailed records of the process by which bills are written and amended, until they finally become law.
Once an initial draft of a new law has been published, parliamentarians (MEPs, for Members of the European Parliament) in one or several specialized committees examine the draft and propose amendments.
Several amendments can be in conflict if they attempt to modify the same part of the law draft.
To be instituted, an amendment needs to be approved by the committee in charge, and ultimately by the full plenary.
The European Parliament publishes every proposed amendment and its authorship, along with various other details.
This makes it possible to build detailed models of the interplay between MEPs, laws, amendments, and committees.

In this work, we (i) curate a large-scale dataset of amendments proposed by MEPs over two legislature periods (2009--2019) and (ii) develop a predictive model for the success and failure of proposed amendments.
Specifically, we collect explicit features for each MEP, including their party membership, country of origin, and gender.
We also collect explicit features of the amendments and dossiers (law drafts), including  their type and the committee in charge.
Finally, we extract the actual text of the amendments, which consists of \emph{edits} of the proposed law.
Our dataset contains \numprint{449493} edits proposed by \numprint{1214} parliamentarians on \numprint{1889} dossiers

Our model relies mostly on the structure of incompatible edits, which can be viewed as a {\em conflict graph} among all edits that target the same law.
We posit a measure of {\em strength} for each parliamentarian, and an edit inherits the strengths of its supporters.
There are two sources of competition in the process.
First, a proposed edit competes with the status quo, because the edit can be rejected in favor of not changing the existing state of a law.
Our model incorporates this by endowing each law with a measure of {\em inertia} that represents the level of controversy of a law.
Second, proposed edits of a law are frequently mutually exclusive, because they overlap and are incompatible.
These edits then compete against each other, as well as against the status quo.

We further include explicit features and text features into the model.
This combination gives rise to models with improved predictive performance and enables us to make predictions for unseen laws.
We also endow our model with a set of latent features for both laws and MEPs, which capture richer interactions between them.
Indeed, it would seem plausible that an MEP might be an expert in one subject matter, but less knowledgeable in another, which would bear upon their effectiveness in promoting a particular amendment.

% TODO: Adapt with new sections.
The remainder of this chapter is structured as follows.
In Section~\ref{sec:dataset}, we state the problem and provide a detailed description of our dataset.
We describe our statistical models in Section~\ref{sec:models}.
We give the results and interpretations of our experiments in Section~\ref{sec:results}.
We describe related work in Section~\ref{sec:relwork} and conclude in Section~\ref{sec:conclusion}.
