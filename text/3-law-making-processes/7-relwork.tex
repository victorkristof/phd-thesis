%! TEX root = ../thesis.tex
\section{Related Work}
\label{sec:relwork}

Amendment analysis was pioneered by \cite{kreppel1999affects}.
The author compares the influence of the Parliament--as an institution rather than individual MEPs--over the Commission during EP3 (1989--1994) and EP4 (1994--1999).
They do so by modeling the acceptance rate of \numprint{500} amendments.
Similar analyses are developed in \cite{tsebelis2001legislative} and \cite{kreppel2002moving} with datasets of, respectively, \numprint{1000} and \numprint{5000} amendments.
Our work introduces a large dataset of more than \numprint{450000} amendments spanning EP7 and EP8.

Predicting the success of edits has been widely studied in the context of Wikipedia \cite{druck2008learning,adler2007content,yardim2018can}.
Similarly, a whole body of literature covers the conflicts between two Wikipedia edits \cite{sumi2011edit,yasseri2012dynamics} and the quantification of controversy of Wikipedia articles \cite{sepehri2012leveraging,rad2012identifying}.
The notion of conflict is, however, different in our setting, where multiple edits can be in conflict at the same time.
In this case, the task of predicting which edit will be accepted out of all the conflicting edits is more complex, and classic approaches cannot be used.

Our model draws inspiration from the discrete choice models.
First, it borrows from the Bradley-Terry model in the pairwise-comparisons literature \cite{zermelo1928berechnung,thurstone1927method,bradley1952rank} to model the competitive dynamics between MEPs.
These approaches learn a real-valued score for individuals and model the probability that one individual wins over another as a function of the difference of their scores.
Second, it borrows from the Rasch model in the item-response theory \cite{rasch1960probabilistic} to model the competitive dynamics between MEPs and the status quo.
These approaches learn a real-valued strength for each individual and a real-valued difficulty for each item, and they model the probability that an individual wins over the item as a function of the difference of the strength and the difficulty.
Our model unifies both approaches by learning a strength for each MEP and a difficulty for each dossier, considering (i) conflicts between MEPs and (ii) conflicts between MEPs and the status quo.
