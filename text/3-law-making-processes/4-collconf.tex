%! TEX root = ../thesis.tex
\section{Edit Graph}
\label{sec:collconf}

We describe the dynamics of the legislative process in terms of the conflicts between edits.
%We take a graph theoretical approach to explain our model in Section~\ref{sec:models}.
For each dossier, we construct the edit graph $ G = (V_G, E_G) $, such that each node $ v \in V_G $ is an edit and such that there is an undirected edge $ (u, v) \in E_G $ if edits $u$ and $v$ overlap.
A component of size at least~2 in $G$ is therefore a group of overlapping edits. % removed ``connected'' -> it's part of the definition of ``component'', so unnecessary
An isolated node corresponds to an edit that does not overlap with any other edit.

In Figure~\ref{fig:edit_graph}, we show the edit graphs of three regulations of EP7.
We depict each node with a green dot if the edit is accepted, and with a red cross if the edit is rejected.
The "transportable pressure equipment" (left), a very specific legislation, exhibits a graph with 96 nodes, among which 97\% are accepted.
The graph contains only isolated nodes, meaning that no edits overlap: all its components are size 1.
The "European capitals of culture" (center), which can affect some cities of member states, exhibits a graph with 58 nodes, among which 48\% are accepted.
The graph contains 16 cliques and the average component size is 1.49.
The GDPR (right), with high stakes for both businesses and consumers, exhibits a graph with 3154 nodes, among which only 9\% are accepted.
The graph contains 1298 cliques, meaning that many edits are conflicting, and has an average component size of 3.44.

Conflicts are inherent in the ordinary legislative procedure defined in Section~\ref{sec:background}, as every proposed edit reflects a disagreement with the initial law proposal.
A first class of conflicts occur between the proposal and each edit proposed by MEPs.
These conflicts appear as components of any size in $G$.
Hence, every isolated node and every clique in $G$ are such conflicts.
We call them "conflicts with the status quo", as they are in disagreement with the proposal.
For example, each edit of Amendments 108 and 5 in Figure~\ref{fig:amendment} is such a conflict.
In Figure~\ref{fig:edit_graph} (left), each green node is an edit accepted over the status quo, and each red node is an edit rejected over the status quo.
Similarly, in Figure~\ref{fig:edit_graph} (center), the cliques with all red nodes are rejected over the status quo.

Another class of conflicts occur between two or more edits proposed by MEPs.
If several MEPs propose different edits on the same part of a text, they compete with each other for the acceptance of their suggestions.
In this case, the edits conflict with the status quo \textit{and} with edits proposed by other MEPs.
These conflicts appear as a clique of size at least~2 in $G$, as there is an edge between overlapping edits.
For example, in Figure~\ref{fig:amendment}, the first edit in Amendment 108 and the first edit in Amendment 5 form such a conflict.
It corresponds to a clique of size~2.
In Figure~\ref{fig:edit_graph} (left), there are no such conflicts.
As no edge links any two nodes, all conflicts are only with the status quo.
In Figure~\ref{fig:edit_graph} (center), however, the cliques with one green node and one or more red nodes are conflicts between several edits, where one edit is accepted over the others and over the status quo.

In $G$, two green nodes cannot appear at both ends of the same edge, as only one edit can be accepted among those that are conflicting.
% Hence, any clique can have at most one green node, and green nodes can only appear as an independent set on the components.
Hence, green nodes can only appear as an independent set on the components.
Two red nodes, however, can appear at both ends of the same edge, as they can both be rejected: this is the case with the first edit in Amendments 108 and 5.
