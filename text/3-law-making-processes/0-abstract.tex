%! TEX root = ../thesis.tex

Similar to a peer-production system, a body of law is an example of a dynamic corpus of text documents that are jointly maintained by a group of editors who compete and collaborate in complex constellations.
Our goal is to develop predictive models for this process, thereby shedding light on the competitive dynamics of parliamentarians who make laws.
For this purpose, we curated a dataset of \numprint{450000} legislative edits introduced by European parliamentarians over the last ten years.
An \textit{edit} modifies the status quo of a law, and could be in competition with another edit if it modifies the same part of that law.
We propose a model for predicting the success of such edits, in the face of both the \textit{inertia} of the status quo and the \textit{competition} between overlapping edits.
The parameters of this model can be interpreted in terms of the influence of parliamentarians and of the controversy of laws.
%We show that parliamentarians in charge of a law under consultation by a legislative committee have influence on the success of edits.
